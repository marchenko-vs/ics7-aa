\chapter{Исследовательская часть}

\section{Технические характеристики устройства}

Технические характеристики устройства, на котором было проведено измерение времени работы алгоритмов:

\begin{enumerate}
\item[1)]
операционная система Windows 11 Pro x64;
\item[2)]
оперативная память 16 ГБ;
\item[3)]
процессор Intel\textregistered ~Core\texttrademark ~i7-4790K @ 4.00 ГГц.
\end{enumerate}

\section{Время работы алгоритмов}

Время работы функций замерены с помощью ассемблерной инструкции \textbf{rdtsc}, которая читает счетчик Time Stamp Counter и возвращает 64-битное количество тиков с момента последнего сброса процессора.

В таблице~\ref{tabular:time} приведено время работы в тиках четырех функций, реализующих алгоритмы поиска редакционного расстояния при различном количестве символов во входных строках. 
На рисунке~\ref{img:graph} изображена зависимость времени работы в тиках алгоритмов поиска редакционного расстояния от количества символов во входных строках.

\begin{table}[H]
\caption{Время работы в тиках алгоритмов поиска редакционного расстояния в зависимости от количества символов во входных строках}
\label{tabular:time}
\begin{tabular}{|>{\raggedleft}p{2.5cm}|>{\raggedleft}p{2.9cm}|>{\raggedleft}p{3.2cm}|>{\raggedleft}p{3.2cm}|>{\raggedleft}p{3.2cm}|}
\hline
\textbf{Кол-во символов} & \textbf{Алгоритм Левенштейна (тики)} & \textbf{Алгоритм Дамерау-Левенштейна (тики)} & \textbf{Алгоритм Дамерау-Левенштейна рекурсивный (тики)} & \textbf{Алгоритм Дамерау-Левенштейна рекурсивный с кэшем (тики)} \tabularnewline
\hline
1 & 6376 & 7060 & 2532 & 12360
\tabularnewline
\hline
2 & 7296 & 8572 & 6200 & 14452
\tabularnewline
\hline
3 & 10012 & 11216 & 18792 & 17988
\tabularnewline
\hline
4 & 10224 & 12612 & 24172 & 28584
\tabularnewline
\hline
5 & 10384 & 13668 & 25272 & 30320
\tabularnewline
\hline
6 & 13636 & 15496 & 30556 & 34984
\tabularnewline
\hline
7 & 136727 & 171927 & 575907 & 36432
\tabularnewline
\hline
8 & 177248 & 175408 & 665208 & 55660
\tabularnewline
\hline
9 & 43902 & 18156 & 120388 & 63300
\tabularnewline
\hline
10 & 45228 & 18308 & 239113 & 114828
\tabularnewline
\hline
\end{tabular}
\end{table}

\includeimage
    {graph}
    {f}
    {H}
    {1\textwidth}
    {Зависимость времени работы в тиках алгоритмов поиска редакционного расстояния в зависимости от количества символов во входных строках}

\section*{Вывод из исследовательской части}

Согласно полученным при проведении эксперимента данным, наиболее эффективным по скорости работы можно считать итерационный алгоритм поиска расстояния Дамерау-Левенштейна. 
Наименее эффективным оказался рекурсивный алгоритм поиска расстояния Дамерау-Левенштейна. 
Его оптимизированная версия с кэшем работает быстрее, но все равно не является такой же быстрой, как итерационные алгоритмы.

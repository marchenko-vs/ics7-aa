\chapter{Конструкторская часть}

\section{Описание алгоритмов}

На рисунках~\ref{img:levenshtein}--\ref{img:damerau_levenshtein_recursive_cache_2} показаны схемы четырех алгоритмов вычисления редакционного расстояния.

\includeimage
    {levenshtein}
    {f}
    {H}
    {1\textwidth}
    {Итерационный алгоритм поиска расстояния Левенштейна}
    
\includeimage
    {damerau_levenshtein_1}
    {f}
    {H}
    {1\textwidth}
    {Итерационный алгоритм поиска расстояния Дамерау-Левенштейна --- ч. 1}
    
\includeimage
    {damerau_levenshtein_2}
    {f}
    {H}
    {1\textwidth}
    {Итерационный алгоритм поиска расстояния Дамерау-Левенштейна --- ч. 2}
    
\includeimage
    {damerau_levenshtein_recursive}
    {f}
    {H}
    {.6\textwidth}
    {Рекурсивный алгоритм поиска расстояния Дамерау-Левенштейна}
    
\includeimage
    {damerau_levenshtein_recursive_cache_1}
    {f}
    {H}
    {.6\textwidth}
    {Рекурсивный алгоритм поиска расстояния Дамерау-Левенштейна с кэшем --- ч. 1}
    
\includeimage
    {damerau_levenshtein_recursive_cache_2}
    {f}
    {H}
    {.6\textwidth}
    {Рекурсивный алгоритм поиска расстояния Дамерау-Левенштейна с кэшем --- ч. 2}

\section{Описание типов данных}

Для реализации алгоритма поиска расстояния Левенштейна и трех модификаций алгоритма поиска расстояния Дамерау-Левенштейна были использованы следующие типы данных:
\begin{enumerate}
\item[1)] строка --- класс \textbf{std::string} из стандартной библиотеки языка \textbf{C\texttt{++}};
\item[2)] длина строки --- переменная типа \textbf{std::size\_t};
\item[3)] матрица --- вектор векторов типа \textbf{std::vector}, состоящий из элементов типа \textbf{int}.
\end{enumerate}

\section{Оценка затрат алгоритмов по памяти}

Итерационные алгоритмы поиска расстояния Левенштейна и Дамерау-Левенштейна не отличаются по использованию памяти.

\subsection{Итерационный алгоритм поиска расстояния Левенштейна (Дамерау-Левенштейна)}

Затраты по памяти для итерационного алгоритма поиска расстояния Левенштейна (Дамерау-Левенштейна).

\begin{enumerate}
\item Длины строк $N$ и $M$: $2 \cdot \mbox{sizeof}(int)$.
\item Строки: $(N + M + 2) \cdot \mbox{sizeof}(char)$.
\item Матрица: $(N + 1) \cdot (M + 1) \cdot \mbox{sizeof}(int) + 2 \cdot \mbox{sizeof}(int)$.
\item Вспомогательные переменные $\mbox{i}, \mbox{j}$: $2 \cdot \mbox{sizeof}(int)$.
\item Адрес возврата.
\end{enumerate}

Суммарные затраты по памяти: 

\begin{equation}
\begin{gathered}
I = (N + 1) \cdot (M + 1) \cdot \mbox{sizeof}(int) + 4 \cdot \mbox{sizeof}(int) + \\ + (N + M + 2) \cdot \mbox{sizeof}(char).
\end{gathered}
\end{equation}

\subsection{Рекурсивный алгоритм поиска расстояния Дамерау-Левенштейна}

Затраты по памяти для рекурсивного алгоритма поиска расстояния
Дамерау-Левенштейна для одного вызова.

\begin{enumerate}
\item Длины строк $N$ и $M$: $2 \cdot \mbox{sizeof}(int)$.
\item Строки: $(N + M + 2) \cdot \mbox{sizeof}(char)$.
\item Вспомогательные переменные $\mbox{i}, \mbox{j}$: $2 \cdot \mbox{sizeof}(int)$.
\item Адрес возврата.
\end{enumerate}

Пусть $R$ --- количество рекурсивных вызовов. 
Тогда суммарные затраты по памяти: 

\begin{equation}
\begin{gathered}
I = R \cdot (4 \cdot \mbox{sizeof}(int) + (N + M + 2) \cdot \mbox{sizeof}(char)).
\end{gathered}
\end{equation}

\subsection{Рекурсивный алгоритм поиска расстояния Дамерау-Левенштейна с кэшем}

Затраты по памяти для рекурсивного алгоритма поиска расстояния Дамерау-Левенштейна с кэшем для одного вызова.

\begin{enumerate}
\item Длины строк $N$ и $M$: $2 \cdot \mbox{sizeof}(int)$.
\item Строки: $(N + M + 2) \cdot \mbox{sizeof}(char)$.
\item Матрица: $(N + 1) \cdot (M + 1) \cdot \mbox{sizeof}(int) + 2 \cdot \mbox{sizeof}(int)$.
\item Вспомогательные переменные $\mbox{i}, \mbox{j}$: $2 \cdot \mbox{sizeof}(int)$.
\item Адрес возврата.
\end{enumerate}

Пусть $R$ --- количество рекурсивных вызовов. 
Тогда суммарные затраты по памяти:

\begin{equation}
\begin{gathered}
I = R \cdot (6 \cdot \mbox{sizeof}(int) + (N + M + 2) \cdot \mbox{sizeof}(char)) + \\ + (N + 1) \cdot (M + 1) \cdot \mbox{sizeof}(int).
\end{gathered}
\end{equation}

\section*{Вывод из конструкторской части}

В текущем разделе на основе теоретических данных, полученных из аналитического раздела, для четырех рассматриваемых в данной лабораторной работе алгоритмов:

\begin{enumerate}
\item[1)] были построены схемы;
\item[2)] были выбраны необходимые для реализации этих алгоритмов типы данных;
\item[3)] была проведена оценка затрачиваемого объема памяти.
\end{enumerate}

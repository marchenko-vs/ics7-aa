\chapter{Аналитическая часть}

\section{Цели и задачи}

Цель работы: получить навыки динамического программирования на материале алгоритмов вычисления редакционного расстояния.

Задачи текущей лабораторной работы:

\begin{enumerate}
\item[1)] изучить понятие редакционного расстояния;
\item[2)] изучить и реализовать четыре алгоритма вычисления редакционного расстояния --- итерационный алгоритм поиска расстояния Левенштейна, итерационный алгоритм поиска расстояния Дамерау-Левенштейна, рекурсивный алгоритм поиска расстояния Дамерау-Левенштейна и рекурсивный алгоритм поиска расстояния Дамерау-Левенштейна с кэшем;
\item[3)] провести сравнительный анализ затрачиваемой памяти всеми алгоритмами на основе теоретических расчетов;
\item[4)]
провести сравнительный анализ времени работы алгоритмов на основе экспериментальных данных.
\end{enumerate}

\section{Итерационный алгоритм поиска расстояния Левенштейна}

Расстояние Левенштейна между двумя строками в теории информации и компьютерной лингвистике --- это минимальное количество операций вставки одного символа, удаления одного символа и замены одного символа на другой, необходимых для превращения одной строки в другую \cite{Leschenko2018}.

Рекуррентная формула, описывающая расстояние Левенштейна:

\begin{equation}
d(S_1, S_2) = D(M, N),~\text{где}
\end{equation}

\begin{equation}
	\label{eq:D}
	D(i, j) = \begin{cases}
		0, &\text{i = 0, j = 0,}\\
		i, &\text{j = 0, i > 0,}\\
		j, &\text{i = 0, j > 0,}\\
		\min \lbrace \\
		\qquad D(i, j-1) + 1\\
		\qquad D(i-1, j) + 1 &\text{i > 0, j > 0.}\\
		\qquad D(i-1, j-1) + m(a[i], b[j]) \\
		\rbrace,
	\end{cases}
\end{equation}

Функция $m(a, b)$ определена как:
\begin{equation}
	\label{eq:m}
	m(a, b) = \begin{cases}
		0, &\text{если a = b,}\\
		1, &\text{иначе}.
	\end{cases}.
\end{equation}

\section{Итерационный алгоритм поиска расстояния Дамерау-Левенштейна}

В алгоритме поиска расстояния Дамерау-Левенштейна добавляется еще одна операция --- транспозиция (перестановка двух соседних символов).

Рекуррентная формула, описывающая расстояние Дамерау-Левенштейна:

\begin{equation}
d(S_1, S_2) = D(M, N),~\text{где}
\end{equation}

\begin{equation}
	\label{eq:a}
	d_{a,b}(i, j) = \begin{cases}
		\max(i, j), \text{ если }\min(i, j) = 0,\\
		\min \lbrace \\
		\qquad d_{a,b}(i, j-1) + 1,\\
		\qquad d_{a,b}(i-1, j) + 1,\\
		\qquad d_{a,b}(i-1, j-1) + m(a[i], b[j]), \text{ иначе}\\
		\qquad \left[ \begin{array}{cc}d_{a,b}(i-2, j-2) + 1, \text{ если }i,j > 1,\\
			\qquad \text{}a[i] = b[j-1], \\
			\qquad \text{}b[j] = a[i-1],\\
			\qquad \infty, \text{ иначе.}\end{array}\right.\\
		\rbrace
	\end{cases}
\end{equation}

\section{Рекурсивный алгоритм поиска расстояния Дамерау-Левенштейна}

Рекурсивный алгоритм поиска расстояния Дамерау-Левенштейна отличается от своей итерационной версии тем, что вместо использования матрицы для хранения вычисленных ранее значений, необходимых для подсчета последующих, эти значения вычисляются каждый раз с помощью рекурсивных вызовов.

\section{Рекурсивный алгоритм поиска расстояния Дамерау-Левенштейна с кэшем}

Данная версия алгоритма поиска расстояния Дамерау-Левенштейна является оптимизацией рекурсивного алгоритма поиска расстояния Дамерау-Левенштейна. 
Оптимизация заключается в использовании кэша, который представляет собой матрицу. 
В нее записываются значения, вычисленные на различных этапах рекурсии. 
Таким образом, при необходимости вычисления некоторого нового значения по реккурентной формуле, величины, необходимые для этого, не вычисляются заново каждый раз. 
Сначала проверяется, было ли уже вычислено данное значение. 
В случае, если этого не происходило ранее, выполняются рекурсивные вычисления для получения нового значения.

\section*{Вывод из аналитической части}

В текущем разделе были рассмотрены четыре алгоритма вычисления редакционного расстояния: итерационный алгоритм поиска расстояния Левенштейна, итерационный алгоритм поиска расстояния Дамерау-Левенштейна, рекурсивный алгоритм поиска расстояния Дамерау-Левенштейна и рекурсивный алгоритм поиска расстояния Дамерау-Левенштейна с кэшем.

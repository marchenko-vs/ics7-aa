{\center\chapter*{Заключение}}
\addcontentsline{toc}{chapter}{Заключение}

В рамках данной лабораторной работы была достигнута поставленная цель: были получены навыки динамического программирования на материале алгоритмов вычисления редакционного расстояния.

Решены все поставленные задачи:

\begin{enumerate}
\item[1)] было изучено понятие редакционного расстояния;
\item[2)] были изучены и реализованы четыре алгоритма вычисления редакционного расстояния --- итерационный алгоритм поиска расстояния Левенштейна, итерационный алгоритм поиска расстояния Дамерау-Левенштейна, рекурсивный алгоритм поиска расстояния Дамерау-Левенштейна и рекурсивный алгоритм поиска расстояния Дамерау-Левенштейна с кэшем;
\item[3)] был проведен сравнительный анализ затрачиваемой памяти всеми алгоритмами на основе теоретических расчетов;
\item[4)]
был проведен сравнительный анализ времени работы алгоритмов на основе экспериментальных данных.
\end{enumerate}

Согласно полученным при проведении эксперимента данным, наиболее эффективными по скорости работы можно считать итерационные алгоритмы поиска расстояния Левенштейна и Дамерау-Левенштейна. 
Менее эффективным является рекурсивный алгоритм поиска расстояния Дамерау-Левенштейна с кэшем. 
Медленее всех работает неоптимизированный рекурсивный алгоритм поиска расстояния Дамерау-Левенштейна.

\chapter{Технологическая часть}

В текущем разделе приведены средства реализации алгоритмов поиска редакционного расстояния и листинги кода.

\section{Требования к программному обеспечению}

Программа должна запрашивать у пользователя две строки и порядковый номер алгоритма поиска редакционного расстояния. 
Затем на экран должен быть выведен результат --- вычисленное редакционное расстояние.
 
\section{Средства реализации}

Для реализации программного обеспечения был выбран язык \textbf{C\texttt{++}} ввиду следующих причин:

\begin{enumerate}
\item[1)]
в библиотеке стандартных шаблонов имеется контейнер \textbf{std::vector}, который можно использовать для создания матриц;
\item[2)]
в стандартной библиотеке есть класс \textbf{std::string};
\item[3)]
для считывания данных и вывода их на экран в стандартной библиотеке существуют соответственно функции \textbf{scanf()} и \textbf{printf()}.
\end{enumerate}

Таким образом, с помощью языка \textbf{C\texttt{++}} можно реализовать программное обеспечение, которое соответствует перечисленным выше требованиям.

\pagebreak
\section{Реализация алгоритмов}

\subsection{Итерационный алгоритм поиска расстояния Левенштейна}

В листинге~\ref{lst:levenshtein.cpp} показана реализация итерационного алгоритма поиска расстояния Левенштейна.

\includelistingpretty
{levenshtein.cpp}{c++}{Реализация итерационного алгоритма поиска расстояния Левенштейна}

\subsection{Итерационный алгоритм поиска расстояния Дамерау-Левенштейна}

В листинге~\ref{lst:damerauLevenshtein.cpp} показана реализация итерационного алгоритма поиска расстояния Дамерау-Левенштейна.

\includelistingpretty
{damerauLevenshtein.cpp}{c++}{Реализация итерационного алгоритма поиска расстояния Дамерау-Левенштейна}

\subsection{Рекурсивный алгоритм поиска расстояния Дамерау-Левенштейна}

В листинге~\ref{lst:damerauLevenshteinRec.cpp} показана реализация рекурсивного алгоритма поиска расстояния Дамерау-Левенштейна.

\includelistingpretty
{damerauLevenshteinRec.cpp}{c++}{Реализация рекурсивного алгоритма поиска расстояния Дамерау-Левенштейна}

\subsection{Рекурсивный алгоритм поиска расстояния Дамерау-Левенштейна с кэшем}

В листинге~\ref{lst:damerauLevenshteinCache.cpp} показана реализация рекурсивного алгоритма поиска расстояния Дамерау-Левенштейна с кэшем.

\includelistingpretty
{damerauLevenshteinCache.cpp}{c++}{Реализация рекурсивного алгоритма поиска расстояния Дамерау-Левенштейна с кэшем}

\section{Тестовые данные}

Пусть $\lambda$ --- пустая строка.  
В таблице \ref{tabular:testsdata} приведены тестовые данные для четырех функций, реализующих алгоритмы поиска редакционного расстояния. 
Тесты выполнялись по методологии черного ящика (модульное тестирование). 
Все тесты пройдены успешно.

\begin{table}[H]
\caption{Тестовые данные}
\label{tabular:testsdata}
\begin{tabular}{|p{4cm}|p{4cm}|p{3.5cm}|p{3.5cm}|}
\hline
\textbf{Первая строка} & \textbf{Вторая строка} & \textbf{Расстояние Левенштейна} & \textbf{Расстояние Дамерау-Левенштейна} \tabularnewline
\hline
$\lambda$ & $\lambda$ & 0 & 0
\tabularnewline
\hline
donut & $\lambda$ & 5 & 5
\tabularnewline
\hline
$\lambda$ & bread & 5 & 5
\tabularnewline
\hline
milk & milk & 0 & 0
\tabularnewline
\hline
rabbit & rabqit & 1 & 1
\tabularnewline
\hline
excited & ecxited & 2 & 1
\tabularnewline
\hline
friend & frien & 1 & 1
\tabularnewline
\hline
\end{tabular}
\end{table}

\section*{Вывод из технологической части}

В данном разделе был написан исходный код четырех алгоритмов поиска редакционного расстояния. 
Описаны тесты и приведены результаты тестирования.

{\center\chapter*{Введение}}
\addcontentsline{toc}{chapter}{Введение}

Динамическое программирование --- метод оптимизации, приспособленный к операциям, в которых процесс принятия решений может быть разбит на отдельные этапы (шаги). 
Такие операции называются многошаговыми \cite{Voytenko1979}.

Алгоритмы поиска и сравнения последовательностей активно используются при работе с неструктурированными данными, обработке больших объемов информации, поисковых запросов и т. д. 
Возросший объем информации предъявляет все более высокие требования к качеству и скорости поиска \cite{Prytkov2012}.

Нечеткий поиск --- это поиск информации, при котором выполняется сопоставление информации заданному образцу поиска или близкому к этому образцу значению. 
Алгоритмы нечеткого поиска используются в большинстве современных поисковых систем (например, для проверки орфографии) \cite{Mosalev2013}.

Задачи нечеткого поиска чаще всего возникают при коррекции ошибок, фильтрации нежелательных сообщений, обнаружении плагиата, поиске с учетом форм одного и того же слова и основаны на определении расстояния между строками. 
Эти методы используются также и в генетике \cite{Prytkov2012}.

В данной лабораторной работе динамическое программирование будет изучено на материале алгоритмов вычисления редакционного расстояния. 
Будут рассмотрены четыре алгоритма: итерационный алгоритм поиска расстояния Левенштейна, итерационный алгоритм поиска расстояния Дамерау-Левенштейна, рекурсивный алгоритм поиска расстояния Дамерау-Левенштейна и рекурсивный алгоритм поиска расстояния Дамерау-Левенштейна с кэшем.

\chapter{Конструкторская часть}
\section{Описание алгоритмов}

На рисунках~\ref{img:reverse_array}--\ref{img:get_min} показаны схемы алгоритмов вспомогательных функций, которые используются в рассматриваемых алгоритмах сортировки. 
На рисунках~\ref{img:cocktail_sort_1}--\ref{img:bucket_sort_4} показаны схемы трех алгоритмов сортировки --- перемешиванием, поразрядной и блочной.

\includeimage
    {reverse_array}
    {f}
    {H}
    {0.8\textwidth}
    {Алгоритм получения массива, в котором элементы находятся в порядке, обратном исходному}

\includeimage
    {get_max}
    {f}
    {H}
    {0.8\textwidth}
    {Алгоритм поиска наибольшего элемента в массиве}
    
\includeimage
    {get_min}
    {f}
    {H}
    {0.8\textwidth}
    {Алгоритм поиска наименьшего элемента в массиве}
    
\includeimage
    {cocktail_sort_1}
    {f}
    {H}
    {0.75\textwidth}
    {Алгоритм сортировки перемешиванием --- ч. 1}
    
\includeimage
    {cocktail_sort_2}
    {f}
    {H}
    {0.8\textwidth}
    {Алгоритм сортировки перемешиванием --- ч. 2}
    
\includeimage
    {counting_sort}
    {f}
    {H}
    {0.9\textwidth}
    {Алгоритм сортировки подсчетом}
    
\includeimage
    {radix_sort_1}
    {f}
    {H}
    {0.75\textwidth}
    {Алгоритм поразрядной сортировки --- ч. 1}
    
\includeimage
    {radix_sort_2}
    {f}
    {H}
    {0.6\textwidth}
    {Алгоритм поразрядной сортировки --- ч. 2}
    
\includeimage
    {radix_sort_3}
    {f}
    {H}
    {0.6\textwidth}
    {Алгоритм поразрядной сортировки --- ч. 3}
    
\includeimage
    {radix_sort_4}
    {f}
    {H}
    {0.45\textwidth}
    {Алгоритм поразрядной сортировки --- ч. 4}
    
\includeimage
    {pos_bucket_sort}
    {f}
    {H}
    {0.9\textwidth}
    {Алгоритм блочной сортировки для неотрицательных чисел}
    
\includeimage
    {bucket_sort_1}
    {f}
    {H}
    {0.75\textwidth}
    {Алгоритм блочной сортировки --- ч. 1}
    
\includeimage
    {bucket_sort_2}
    {f}
    {H}
    {0.65\textwidth}
    {Алгоритм блочной сортировки --- ч. 2}
    
\includeimage
    {bucket_sort_3}
    {f}
    {H}
    {0.7\textwidth}
    {Алгоритм блочной сортировки --- ч. 3}
    
\includeimage
    {bucket_sort_4}
    {f}
    {H}
    {0.45\textwidth}
    {Алгоритм блочной сортировки --- ч. 4}  

\section{Модель вычислений}

Обозначим размер массива во всех вычислениях через $N$.

Для вычисления трудоемкости введена следующая модель вычислений.

\begin{enumerate}
\item
Операции +, --, *, /, \%, ==, !=, <, <=, >, >= [~], ++, -- --, +=, --= имеют трудоемкость 1.

\item
Трудоемкость условного оператора рассчитывается следующим образом:
\begin{equation}
f_{if} = f_{\text{условия}} +
	\begin{cases}
    f_A,~\text{если условие выполняется}, \\
    f_B,~\text{иначе}.
    \end{cases}
\end{equation}

\item
Трудоемкость цикла рассчитывается как:
\begin{equation}
f_{loop} = f_{\text{инициализации}} + f_{\text{сравнения}} + N \cdot (f_{\text{тела}} + f_{\text{инкремента}} +  f_{\text{сравнения}}).
\end{equation}

\item
Трудоемкость вызова функции равна 0.
\end{enumerate}

\section{Трудоемкость алгоритмов}

\subsection{Алгоритм сортировки перемешиванием}
\label{sub:cocktail_sort}

Сравнение для проверки количества элементов в массиве:
\begin{equation}
f_1 = 1.
\end{equation}

Блок инициализации:
\begin{equation}
f_2 = 4.
\end{equation}

Основной цикл:
\begin{equation}
\begin{gathered}
f_3 = 1 + N \cdot ((1 + 2 + N \cdot (5 + 4 + 1 + 2) + 1 + 2) + \\ + 3 + N \cdot (5 + 4 + 1) + 1)) = \\ = 1 + N \cdot ((6 + 12 \cdot N) + 4 + 10 \cdot N) = \\ = 1 + N \cdot (10 + 22 \cdot N) = \\ = 1 + 10 \cdot N + 22 \cdot N^2.
\end{gathered}
\end{equation}

Таким образом, трудоемкость алгоритма сортировки перемешиванием равна:
\begin{equation}
\begin{gathered}
f = f_1 + f_2 + f_3 = 1 + 4 + 1 + 10 \cdot N + 22 \cdot N^2 = \\ = 6 + 10 \cdot N + 22 \cdot N^2 \approx O(N^2).
\end{gathered}
\end{equation}

\subsection{Алгоритм поразрядной сортировки}

Обозначим через $M$ количество десятичных разрядов наибольшего элемента в массиве.

Сравнение для проверки количества элементов в массиве:
\begin{equation}
f_1 = 1.
\end{equation}

Блок инициализации:
\begin{equation}
f_2 = 2.
\end{equation}

Первый цикл:
\begin{equation}
f_3 = 2 + 4 \cdot N.
\end{equation}

Второй блок инициализации:
\begin{equation}
f_4 = 3.
\end{equation}

Функция, возвращающая массив, элементы которого находятся в порядке, обратном исходному:
\begin{equation}
f_5 = 1 + 2 + 2 + 4 \cdot \frac{N}{4} = 5 + N.
\end{equation}

Цикл для работы с отрицательными элементами входного массива:
\begin{equation}
\begin{gathered}
f_6 = 2 + 3 + 8 \cdot N + 1 + 2 + 2 + 6 \cdot (N - 1) + \\ + 3 + M \cdot (1 + 2 + 2 + 8 \cdot \frac{N}{2} + 2 + 9 \cdot 7 + 3 + \\ + 11 \cdot \frac{N}{2} + 2 + 5 \cdot \frac{N}{2} + 3) + 5 + N = \\ = 12 + 15 \cdot N + M \cdot (78 + 12 \cdot N) = \\ = 12 + 15 \cdot N + 78 \cdot M + 12 \cdot N \cdot M.
\end{gathered}
\end{equation}

Цикл для работы с неотрицательными элементами входного массива:
\begin{equation}
\begin{gathered}
f_7 = 2 + 3 + 7 \cdot N + 1 + 2 + 2 + 6 \cdot (N - 1) + \\ + 3 + M \cdot (1 + 2 + 2 + 8 \cdot \frac{N}{2} + 2 + 9 \cdot 7 + 3 + \\ + 11 \cdot \frac{N}{2} + 2 + 5 \cdot \frac{N}{2} + 3) = \\ = 12 + 14 \cdot N + 78 \cdot M + 12 \cdot N \cdot M.
\end{gathered}
\end{equation}

Цикл для записи отрицательных элементов в исходный массив:
\begin{equation}
f_8 = 3 + 7 \cdot \frac{N}{2}.
\end{equation}

Цикл для записи неотрицательных элементов в исходный массив:
\begin{equation}
f_9 = 3 + 6 \cdot \frac{N}{2}.
\end{equation}

Таким образом, трудоемкость алгоритма поразрядной сортировки равна:
\begin{equation}
\begin{gathered}
f = f_1 + f_2 + f_3 + f_4 + f_5 + f_6 + f_7 + f_8 + f_9 = \\ = 43 + 40.5 \cdot N + 156 \cdot M + 24 \cdot N \cdot M \approx \\ \approx O(N \cdot M).
\end{gathered}
\end{equation}

\subsection{Алгоритм блочной сортировки}

Обозначим через $K$ количество сегментов, используемых для сортировки элементов.

Сравнение для проверки количества элементов в массиве:
\begin{equation}
f_1 = 1.
\end{equation}

Блок инициализации:
\begin{equation}
f_2 = 2.
\end{equation}

Подсчет количества отрицательных и неотрицательных элементов в исходном массиве:
\begin{equation}
f_3 = 2 + 5 \cdot N.
\end{equation}

Второй блок инициализации:
\begin{equation}
f_4 = 3.
\end{equation}

Функция для поиска наибольшего элемента в массиве:
\begin{equation}
f_5 = 2 + 2 + 6 \cdot (\frac{N}{2} - 1) = 4 + 6 \cdot \frac{N}{2} - 6 = 3 \cdot N - 2.
\end{equation}

Функция для поиска наименьшего элемента в массиве:
\begin{equation}
f_6 = 2 + 2 + 6 \cdot (\frac{N}{2} - 1) = 4 + 6 \cdot \frac{N}{2} - 6 = 3 \cdot N - 2.
\end{equation}

Функция, возвращающая массив, элементы которого находятся в порядке, обратном исходному:
\begin{equation}
f_7 = 3 + 2 + 6 \cdot \frac{N}{4} = 5 + 3 \cdot \frac{N}{2} = 5 + 1.5 \cdot N.
\end{equation}

Функция, реализующая алгоритм сортировки перемешиванием (значение получено в пункте \ref{sub:cocktail_sort}):
\begin{equation}
\begin{gathered}
f_8 = N^2.
\end{gathered}
\end{equation}

Функция для сортировки массива, состоящего из неотрицательных элементов:
\begin{equation}
\begin{gathered}
f_9 = 14 + f_5 + f_6 + 9 \cdot \frac{N}{2} + (5 + f_8) \cdot K + (4 + 9 \cdot \frac{N}{2 \cdot K}) \cdot K = \\ = 14 + 3 \cdot N - 2 + 3 \cdot N - 2 + 4.5 \cdot N + 5 \cdot K + \\ + 18 \cdot K - 4 \cdot N \cdot K + 2 \cdot N^2 \cdot K + 4 \cdot K + 4.5 \cdot N = \\ = 22 \cdot K + 4.5 \cdot N - 4 \cdot N \cdot K + 2 \cdot N^2 \cdot K.
\end{gathered}
\end{equation}

Сортировка массива, состоящего из отрицательных элементов:
\begin{equation}
f_{10} = 6 + 9 \cdot \frac{N}{2} + f_9 + f_7 = 6 + 4.5 \cdot N + f_7 + f_9.
\end{equation}

Сортировка массива, состоящего из неотрицательных элементов:
\begin{equation}
f_{11} = 6 + 8 \cdot \frac{N}{2} + f_9 = 6 + 4 \cdot N + f_9.
\end{equation}

Цикл для записи отрицательных элементов в исходный массив:
\begin{equation}
f_{12} = 3 + 7 \cdot \frac{N}{2} = 3 + 3.5 \cdot N.
\end{equation}

Цикл для записи неотрицательных элементов в исходный массив:
\begin{equation}
f_{13} = 3 + 6 \cdot \frac{N}{2} = 3 + 3 \cdot N.
\end{equation}

Таким образом, трудоемкость алгоритма блочной сортировки равна:
\begin{equation}
\begin{gathered}
f = f_1 + f_2 + f_3 + f_4 + f_{10} + f_{11} + f_{12} + f_{13} = \\ = 8 + 5 \cdot N + 6 + 4.5 \cdot N + f_7 + f_9 + 6 + 4 \cdot N + \\ + f_9 + 3 + 3.5 \cdot N + 3 + 3 \cdot N = \\ = 26 + 20 \cdot N + f_7 + 2 \cdot f_9 = \\ = 26 + 20 \cdot N + 5 + 1.5 \cdot N + 2 \cdot (22 \cdot K + \\ + 4.5 \cdot N - 4 \cdot N \cdot K + 2 \cdot N^2 \cdot K) = \\ = 75 + 30.5 \cdot N - 8 \cdot N \cdot K + 2 \cdot N^2 \cdot K \approx \\ \approx O(N^2 \cdot K).
\end{gathered}
\end{equation}

\section*{Вывод}

В текущем разделе на основе теоретических данных, полученных из аналитического раздела, были построены схемы трех алгоритмов сортировки, а также оценены их трудоемкости.

{\center\chapter*{Введение}}
\addcontentsline{toc}{chapter}{Введение}

Люди встречаются с отсортированными объектами в телефонных книгах, в списках подоходных налогов, в оглавлениях книг, в библиотеках, в словарях, на складах –-- почти везде, где нужно искать хранимые объекты. 
Сортировку следует понимать как процесс перегруппировки заданного множества объектов в некотором определенном порядке. 
Цель сортировки -–- облегчить последующий поиск элементов в таком упорядоченном множестве \cite{Wirth1989}.

Задача сортировки формально определяется следующим образом.

Вход: последовательность из $n$ чисел $\langle a_1, a_2,...,a_n \rangle$.

Выход: перестановка (изменение порядка) $\langle a_1', a_2',...,a_n' \rangle$ входной последовательности таким образом, что для ее членов выполняется соотношение $a_1' \leq a_2'\leq ... \leq a_n'$ \cite{Cormen2011}.

Существует множество различных алгоритмов сортировки. 
Для их сравнения вводится понятие трудоемкости. 
Под трудоемкостью алгоритма для данного конкретного входа в данной модели вычислений понимается количество <<элементарных>> операций, совершаемых алгоритмом.

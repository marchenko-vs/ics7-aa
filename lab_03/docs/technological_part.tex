\chapter{Технологическая часть}

В текущем разделе приведены средства реализации алгоритмов сортировки и листинги кода.

\section{Требования к программному обеспечению}

Программа должна запрашивать размер $N$ массива $A$, а затем сами элементы вектора (все элементы должны быть целыми числами). 
Далее программа должна запросить у пользователя порядковый номер одного из трех алгоритмов сортировки. 
После этого массив должен быть упорядочен по возрастанию с помощью выбранного алгоритма сортировки и выведен на экран.

Программа должна обрабатывать ошибки (например, попытку ввода отрицательного размера массива) и завершать работу с выводом информации об ошибке на экран.
 
\section{Средства реализации}

Для реализации программного обеспечения был выбран язык \textbf{C++} ввиду следующих причин:

\begin{enumerate}
\item[1)]
в библиотеке стандартных шаблонов имеется контейнер \textbf{std::vector};
\item[2)]
для \textbf{std::vector} реализован метод \textbf{size()}, который возвращает размер вектора;
\item[3)]
для считывания данных и вывода их на экран в стандартной библиотеке существуют соответственно функции \textbf{scanf()} и \textbf{printf()}.
\end{enumerate}

Таким образом, с помощью языка \textbf{C++} можно реализовать программное обеспечение, которое соответствует перечисленным выше требованиям.

\section{Реализация алгоритмов}

\subsection{Вспомогательные функции}

В листинге~\ref{lst:additional_functions.c} показана реализация вспомогательных функций, которые нужны для реализации алгоритмов сортировки.

\includelistingpretty
{additional_functions.c}{c++}{Вспомогательные функции для реализации алгоритмов сортировки}

\subsection{Алгоритм сортировки перемешиванием}

В листинге~\ref{lst:cocktail_sort.c} показана реализация алгоритма сортировки перемешиванием.

\includelistingpretty
{cocktail_sort.c}{c++}{Реализация сортировки перемешиванием}

\subsection{Алгоритм поразрядной сортировки}

В листинге~\ref{lst:radix_sort.c} показана реализация алгоритма поразрядной сортировки с использованием сортировки подсчетом.

\includelistingpretty
{radix_sort.c}{c++}{Реализация поразрядной сортировки}

\subsection{Алгоритм блочной сортировки}

В листинге~\ref{lst:bucket_sort.c} показана реализация алгоритма блочной сортировки с использованием сортировки перемешиванием.

\includelistingpretty
{bucket_sort.c}{c++}{Реализация блочной сортировки}

\section{Тестовые данные}

В таблице~\ref{tabular:testsdata} приведены тестовые данные и их описание для трех функций, реализующих алгоритмы сортировки. 
Тесты выполнялись по методологии черного ящика (модульное тестирование). 
Все тесты пройдены успешно.

\begin{table}[H]
\caption{Тестовые данные}
\label{tabular:testsdata}
\begin{tabular}{|p{3.5cm}|p{4cm}|p{4cm}|p{4cm}|}
\hline
\textbf{Описание теста} & \textbf{Входной массив} & \textbf{Ожидаемый результат} & \textbf{Выходной массив} \tabularnewline
\hline
Пустой массив & NULL & NULL & NULL \tabularnewline
\hline
Один отрицательный элемент & -25 & -25 & -25 \tabularnewline
\hline
Один положительный элемент & 76 & 76 & 76 \tabularnewline
\hline
Упорядоченный по возрастанию массив & -282, -50, 32, 54, 76, 108 & -282, -50, 32, 54, 76, 108 & -282, -50, 32, 54, 76, 108 \tabularnewline
\hline
Упорядоченный по убыванию массив & 982, 654, 54, 3, -19, -89, -320 & -320, -89, -19, 3, 54, 654, 982 & -320, -89, -19, 3, 54, 654, 982 \tabularnewline
\hline
Случайно упорядоченный массив & 10, -20, 32, -89, -76, -10, 89, 35, 197 & -89, -76, -20, -10, 10, 32, 35, 89, 197 & -89, -76, -20, -10, 10, 32, 35, 89, 197 \tabularnewline
\hline
\end{tabular}
\end{table}

\section*{Вывод}

В данном разделе был написан исходный код трех алгоритмов сортировки --- перемешиванием, поразрядной и блочной. Описаны тесты и приведены результаты тестирования.

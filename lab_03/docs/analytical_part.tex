\chapter{Аналитическая часть}

\section{Цели и задачи}

Цель работы: получить навыки оценки трудоемкости и временной эффективности на материале алгоритмов сортировки.

Задачи текущей лабораторной работы:

\begin{enumerate}
\item[1)]
изучить и реализовать три алгоритма сортировки --- перемешиванием, поразрядную и блочную;
\item[2)]
провести сравнительный анализ трудоемкости алгоритмов на основе теоретических расчетов и выбранной модели вычислений;
\item[3)]
провести сравнительный анализ алгоритмов на основе экспериментальных данных.
\end{enumerate}

\section{Блочная сортировка}

Блочная сортировка (англ. bucket sort) --- это алгоритм сортировки, который работает путем распределения элементов массива по нескольким сегментам (<<ведрам>>, блокам). 
Каждый блок сортируется индивидуально с использованием любого другого алгоритма сортировки. 
Очевидно, что трудоемкость блочной сортировки зависит от алгоритма, используемого для сортировки каждого сегмента.

Блочная сортировка работает следующим образом.

\begin{enumerate}
\item
Инициализируется массив изначально пустых сегментов.
\item
Происходит обход исходного массива, и каждый элемент помещается в соответствующий блок.
\item
Элементы в каждом непустом сегменте сортируются.
\item
Происходит обход блоков, и все элементы копируются в исходный массив.
\end{enumerate}

\section{Поразрядная сортировка}

Поразрядная сортировка (англ. radix sort) --- алгоритм сортировки, который выполняется за линейное время. 
Он позволяет избежать сравнений, создавая и распределяя элементы по <<корзинам>> в соответствии с их разрядом. 
Для элементов с более чем одной значащей цифрой этот процесс группирования повторяется для каждой цифры, сохраняя порядок предыдущего шага, пока не будут учтены все цифры.

Поразрядная сортировка может применяться к данным, которые можно отсортировать лексикографически, будь то целые числа, слова, перфокарты, игральные карты или почта \cite{Sinha2003}.

\section{Сортировка перемешиванием}

Сортировка перемешиванием (англ. cocktail sort, shaker sort, shuffle sort) --- разновидность пузырьковой сортировки. 
Анализируя метод пузырьковой сортировки, можно отметить два обстоятельства. 
Во-первых, если при обходе части массива перестановки не происходят, то эта часть массива уже отсортирована и, следовательно, ее можно исключить из рассмотрения. 
Во-вторых, при обходе от конца массива к началу минимальный элемент <<всплывает>> на первую позицию, а максимальный элемент сдвигается только на одну позицию вправо. 
Эти две идеи приводят к следующим модификациям в методе пузырьковой сортировки: границы рабочей части массива (обход которой происходит в текущий момент) устанавливаются в месте последнего обмена на каждой итерации и массив обходится поочередно справа налево и слева направо.

\section*{Вывод}

В текущем разделе были рассмотрены три алгоритма сортировки: перемешиванием, поразрядная и блочная.

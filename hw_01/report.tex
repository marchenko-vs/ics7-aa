\documentclass{bmstu}

\newcommand{\titlepageteachers}[7]
{
	{
		\small
		\begin{tabularx}{\textwidth}{@{}>{\hsize=.5\hsize}X>{\hsize=.25\hsize}X>{\hsize=.25\hsize}X@{}}
			\titlepagestudent{#1}{#2}
			\titlepageothers{#3}{#4}
			\titlepageothers{#5}{#6}
		\end{tabularx}
	}
}

\renewcommand{\makereporttitle}[9]
{
	\documentmeta{Отчет}{#8}{#3 #4}{#5}
	
	\begin{titlepage}
		\centering
		
		\titlepageheader{#1}{#2}
		\vspace{15.8mm}
		
		\titlepagereporttitle{#3}{#4}{#5}{#6}
		\vfill
		
		\titlepageteachers{#7}{#8}{Преподаватель}{#9}{Преподаватель}{Л.~Л.~Волкова}{}
		
		\vspace{14mm}
		
		\textit{{\the\year} г.}
	\end{titlepage}
	
	\setcounter{page}{2}
}


\begin{document}

\makereporttitle
    {Информатика и системы управления}
    {Программное обеспечение ЭВМ и информационные технологии}
    {домашнему заданию №~1}
    {Анализ алгоритмов}
    {Параллельные вычисления}
    {}
    {ИУ7-53Б}
    {В.~Марченко}
    {Ю.~В.~Строганов}
    {}

\maketableofcontents

\chapter{Реализация алгоритма}

В листинге~\ref{lst:merge.c} показана реализация алгоритма слияния двух массивов для сортировки слиянием.

\includelistingpretty
{merge.c}{c}{Реализация алгоритма слияния двух массивов}

\chapter{Операционный граф}

На рисунке~\ref{img:graph_01} показан операционный граф.

\includeimage
    {graph_01}
    {f}
    {H}
    {1\textwidth}
    {Операционный граф}

\chapter{Информационный граф}

На рисунке~\ref{img:graph_02} показан информационный граф.

\includeimage
    {graph_02}
    {f}
    {H}
    {1\textwidth}
    {Информационный граф}

\chapter{Граф операционной истории}

На рисунке~\ref{img:graph_03} показан граф операционной истории.

\includeimage
    {graph_03}
    {f}
    {H}
    {1\textwidth}
    {Граф операционной истории}

\chapter{Граф информационной истории}

На рисунке~\ref{img:graph_04} показан граф информационной истории.

\includeimage
    {graph_04}
    {f}
    {H}
    {1\textwidth}
    {Граф информационной истории}

\end{document}

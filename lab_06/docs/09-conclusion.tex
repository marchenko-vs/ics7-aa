{\centering \chapter*{ЗАКЛЮЧЕНИЕ}}
\addcontentsline{toc}{chapter}{ЗАКЛЮЧЕНИЕ}

В ходе выполнения данной лабораторной работы была достигнута поставленная цель: был получен навык поиска по словарю при ограничении на значения признака, заданном при помощи лингвистической переменной.

Решены все поставленные задачи:
\begin{enumerate}
\item[1)] формализован объект и его признак;
\item[2)] составлена анкету для ее заполнения респондентом;
\item[3)] проведено анкетирование респондентов;
\item[4)] построена функция принадлежности термам числовых значений признака, описываемого лингвистической переменной, на основе статистической обработки мнений респондентов, выступающих в роли экспертов;
\item[5)] описаны 3--5 типовых вопросов на русском языке, имеющих целью запрос на поиск в словаре;
\item[6)] описан алгоритм поиска в словаре объектов, удовлетворяющих ограничению, заданному в вопросе на ограниченном ествественном языке;
\item[7)] описана структура данных словаря, хранящего наименования объектов согласно варианту и числовое значение признака объекта;
\item[8)] реализован описанный алгоритм поиска в словаре;
\item[9)] приведены примеры запросов пользователя и сформированной реализациией алгоритма поиска выборки объектов из словаря;
\item[10)] дано заключение о применимости предложенного алгоритма и о его ограничениях.
\end{enumerate}

В ходе выполнения текущей лабораторной работы было реализовано простое вопросно-ответное программное обеспечение, которое имеет ограничение на один объект и его единственный признак.

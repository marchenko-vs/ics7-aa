\chapter{Исследовательская часть}

\section*{Анкетирование респондентов}
\addcontentsline{toc}{section}{Анкетирование респондентов}

Пронумеруем все термы:
\begin{enumerate}
\item[1)] очень короткая;
\item[2)] короткая;
\item[3)] не очень короткая;
\item[4)] средняя;
\item[5)] не очень длинная;
\item[6)] длинная;
\item[7)] очень длинная.
\end{enumerate}

В таблице~\ref{tabular:survey} приведен результат анкетирования респондентов.

\begin{table}[H]
\caption{Результат анкетирования респондентов}
\label{tabular:survey}
\begin{tabular}{|>{\raggedleft}p{3cm}|>{\raggedleft}p{1.4cm}|>{\raggedleft}p{1.4cm}|>{\raggedleft}p{1.4cm}|>{\raggedleft}p{1.4cm}|>{\raggedleft}p{1.4cm}|>{\raggedleft}p{1.4cm}|>{\raggedleft}p{1.4cm}|}
\hline
\textbf{Респондент} & \textbf{1} & \textbf{2} & \textbf{3} & \textbf{4} & \textbf{5} & \textbf{6} & \textbf{7}
\tabularnewline
\hline
Науменко & 1000 & 3000 & 5000--7000 & 7000--9000 & 7000--9000 & 9000 & 11000
\tabularnewline
\hline
Светличная & 1000 & 3000--5000 & 5000 & 7000 & 7000--9000 & 9000--11000 & 11000
\tabularnewline
\hline
Калашников & 1000--3000 & 3000 & 5000 & 5000--7000 & 7000 & 7000--9000 & 9000--11000
\tabularnewline
\hline
Дыхал & 1000 & 3000 & 5000--7000 & 7000--9000 & 9000 & 9000--11000 & 11000
\tabularnewline
\hline
\end{tabular}
\end{table}

В таблице~\ref{tabular:survey} приведена принадлежность термам числовых значений на основе статистической обработки мнений респондентов, выступающих в роли экспертов.

\begin{table}[H]
\caption{Принадлежность термам числовых значений}
\label{tabular:time}
\begin{tabular}{|>{\raggedleft}p{4.5cm}|>{\raggedleft}p{1.4cm}|>{\raggedleft}p{1.4cm}|>{\raggedleft}p{1.4cm}|>{\raggedleft}p{1.5cm}|>{\raggedleft}p{1.5cm}|>{\raggedleft}p{1.5cm}|}
\hline
\textbf{Терм} & \textbf{1000} & \textbf{3000} & \textbf{5000} & \textbf{7000} & \textbf{9000} & \textbf{11000}
\tabularnewline
\hline
Очень короткая & 1 & 0.25 & 0 & 0 & 0 & 0
\tabularnewline
\hline
Короткая & 0 & 1 & 0.25 & 0 & 0 & 0
\tabularnewline
\hline
Не очень короткая & 0 & 0 & 1 & 0.5 & 0 & 0
\tabularnewline
\hline
Средняя & 0 & 0 & 0.25 & 1 & 0.5 & 0
\tabularnewline
\hline
Не очень длинная & 0 & 0 & 0 & 0.75 & 0.75 & 0
\tabularnewline
\hline
Длинная & 0 & 0 & 0 & 0.25 & 1 & 0.5
\tabularnewline
\hline
Очень длинная & 0 & 0 & 0 & 0 & 0.25 & 1
\tabularnewline
\hline
\end{tabular}
\end{table}

На рисунке~\ref{img:graph} изображена принадлежность термам числовых значений на основе статистической обработки мнений респондентов, выступающих в роли экспертов.

\includeimage
    {graph}
    {f}
    {H}
    {1\textwidth}
    {Принадлежность термам числовых значений}

\section*{Вывод из исследовательской части}

В текущем разделе было проведено анкетирование респондентов. 
По результатам опроса была построена функция принадлежности термам числовых значений признака, описываемого лингвистической переменной, на основе статистической обработки мнений респондентов, выступающих в роли экспертов.

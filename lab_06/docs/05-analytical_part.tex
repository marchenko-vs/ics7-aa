\chapter{Аналитическая часть}

\section{Цель и задачи}

Цель работы: получить навык поиска по словарю при ограничении на значения признака, заданном при помощи лингвистической переменной.

Задачи лабораторной работы:
\begin{enumerate}
\item[1)] формализовать объект и его признак;
\item[2)] составить анкету для ее заполнения респондентом;
\item[3)] провести анкетирование респондентов;
\item[4)] построить функцию принадлежности термам числовых значений признака, описываемого лингвистической переменной, на основе статистической обработки мнений респондентов, выступающих в роли экспертов;
\item[5)] описать 3--5 типовых вопросов на русском языке, имеющих целью запрос на поиск в словаре;
\item[6)] описать алгоритм поиска в словаре объектов, удовлетворяющих ограничению, заданному в вопросе на ограниченном ествественном языке;
\item[7)] описать структуру данных словаря, хранящего наименования объектов согласно варианту и числовое значение признака объекта;
\item[8)] реализовать описанный алгоритм поиска в словаре;
\item[9)] привести примеры запросов пользователя и сформированной реализациией алгоритма поиска выборки объектов из словаря, используя составленные респондентами вопросы;
\item[10)] дать заключение о применимости предложенного алгоритма и о его ограничениях.
\end{enumerate}

\section{Словарь}

Словарь --- абстрактный тип данных, позволяющий хранить пары вида <<ключ-значение>> и поддерживающий операции добавления, поиска и удаления пары по ключу. 
В паре $(key,~value)$ значение $value$ называется значением, ассоциированным с ключом $key$. 
Поиск --- основная задача при использовании словаря, которая может решаться различными способами.

\section{Объект и его признак}

Объектами в текущей лабораторной работе являются гоночные трассы Формулы 1. 
Признаком является длина трассы, которая в рамках данной задачи измеряется в метрах. 
Словарь используется для описания обьекта <<гоночная трассы Формулы 1>> со следующими параметрами: ключ --- терм (словесное описание признака), значение --- числовые значения признака (длина трассы в метрах). 
Доступные термы:
\begin{enumerate}
\item[1)] очень короткая;
\item[2)] короткая;
\item[3)] не очень короткая;
\item[4)] средняя;
\item[5)] не очень длинная;
\item[6)] длинная;
\item[7)] очень длинная.
\end{enumerate}

Доступные числовые значения признака: от 1000 метров до 11000 метров.

\section{Вопросы}

Программное обеспечение должно будет отвечать на следюущие вопросы.
\begin{enumerate}
\item Какие трассы Формулы 1 являются длинными?
\item Можешь перечислить все не очень длинные трассы Формулы 1?
\item Какая протяженность у очень длинных трасс Формулы 1?
\item Можешь вывести все трассы Формулы 1?
\end{enumerate}

\section*{Вывод из аналитической части}

В текущем разделе была рассмотрена задача поиска в словаре, формализован объект и его признак, а также перечислены вопросы, на которые должно будет отвечать программное обеспечение. 

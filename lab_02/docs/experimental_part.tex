\chapter{Исследовательская часть}
\section{Технические характеристики устройства}

Технические характеристики устройства, на котором было проведено измерение времени работы алгоритмов:

\begin{enumerate}
\item[1)]
операционная система Windows 10 Pro x64;
\item[2)]
оперативная память 8 ГБ;
\item[3)]
процессор Intel\textregistered ~Core\texttrademark ~i5-11300H @ 3.10 ГГц.
\end{enumerate}

\section{Время работы алгоритмов}

Время работы функций замерены с помощью ассемблерной инструкции \textbf{rdtsc}, которая читает счетчик Time Stamp Counter и возвращает 64-битное количество тиков с момента последнего сброса процессора.

В таблице~\ref{tabular:time} приведено время работы в тиках трех функций, реализующих алгоритмы умножения матриц при различном количестве элементов во входных матрицах (обе входные матрицы квадратные и одинакового порядка). На рисунке~\ref{img:graph} изображена зависимость времени работы в тиках алгоритмов умножения матриц от количества элементов во входных матрицах.

\begin{table}[H]
\caption{Время работы в тиках алгоритмов умножения матриц в зависимости от количества элементов во входных матрицах}
\label{tabular:time}
\begin{tabular}{|>{\raggedleft}p{3.5cm}|>{\raggedleft}p{4cm}|>{\raggedleft}p{4cm}|>{\raggedleft}p{4cm}|}
\hline
\textbf{Количество элементов в матрицах} & \textbf{Классический алгоритм (тики)} & \textbf{Алгоритм Копперсмита--Винограда (тики)} & \textbf{Алгоритм Копперсмита--Винограда оптимизированный (тики)} \tabularnewline
\hline
4 & 22808 & 19768 & 18861 \tabularnewline
\hline
9 & 28537 & 32882 & 31529 \tabularnewline
\hline
16 & 40044 & 48256 & 40338 \tabularnewline
\hline
25 & 63844 & 77181 & 74470 \tabularnewline
\hline
36 & 85947 & 153370 & 165253 \tabularnewline
\hline
49 & 123067 & 154453 & 168738 \tabularnewline
\hline
64 & 134918 & 179822 & 171832 \tabularnewline
\hline
81 & 150907 & 186049 & 187017 \tabularnewline
\hline
100 & 202561 & 209814 & 209873 \tabularnewline
\hline
121 & 209775 & 226675 & 215758 \tabularnewline
\hline
144 & 226007 & 263103 & 246680 \tabularnewline
\hline
169 & 237456 & 377059 & 287474 \tabularnewline
\hline
196 & 314170 & 391086 & 333768 \tabularnewline
\hline
225 & 318235 & 490527 & 437600 \tabularnewline
\hline
256 & 343095 & 661921 & 577530 \tabularnewline
\hline
289 & 524986 & 667464 & 599127 \tabularnewline
\hline
324 & 549017 & 755572 & 609720 \tabularnewline
\hline
361 & 562894 & 771701 & 658774 \tabularnewline
\hline
400 & 599735 & 850421 & 670344 \tabularnewline
\hline
441 & 711790 & 896018 & 702033 \tabularnewline
\hline
\end{tabular}
\end{table}

\includeimage
    {graph}
    {f}
    {H}
    {1\textwidth}
    {Зависимость времени работы в тиках алгоритмов умножения матриц от количества элементов во входных матрицах}

\section*{Вывод из исследовательской части}

Согласно полученным при проведении эксперимента данным, наиболее эффективным можно считать классический алгоритм умножения матриц. На второе место по скорости работы можно поставить оптимизированный алгоритм Копперсмита--Винограда. Наименее эффективным оказался обычный алгоритм Копперсмита--Винограда.

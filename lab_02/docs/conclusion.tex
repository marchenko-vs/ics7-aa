{\center\chapter*{Заключение}}
\addcontentsline{toc}{chapter}{Заключение}

В рамках данной лабораторной работы была достигнута поставленная цель: были получены навыки оценки трудоемкости и временной эффективности на материале алгоритмов умножения матриц.

Решены все поставленные задачи:

\begin{enumerate}
\item[1)]
были изучены и реализованы три алгоритма умножения матриц --- классический, Копперсмита--Винограда и оптимизированный алгоритм Копперсмита--Винограда;
\item[2)]
был проведен сравнительный анализ трудоемкости алгоритмов на основе теоретических расчетов и выбранной модели вычислений;
\item[3)]
был проведен сравнительный анализ времени работы алгоритмов на основе экспериментальных данных.
\end{enumerate}

Согласно полученным при проведении эксперимента данным, самым эффективным по времени работы является классический алгоритм умножения матриц, затем оптимизированный алгоритм Копперсмита--Винограда и обычный алгоритм Копперсмита--Винограда.

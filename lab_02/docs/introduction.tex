{\center\chapter*{Введение}}
\addcontentsline{toc}{chapter}{Введение}

В настоящее время матричное исчисление широко применяется в различных областях математики, механики, теоретической физики, теоретической электротехники и т. д. \cite{Gantmacher1966}

Один из примеров использования матриц --- аффинные преобразования в пространстве (или на плоскости) в компьютерной графике (параллельный перенос, поворот и масштабирование). Если объект нужно преобразовать, то нужно использовать операцию умножения матриц для всех вершин этого тела.

В математике умножение матриц --- это бинарная операция, которая создает матрицу из двух исходных. Для умножения матриц количество столбцов в первой матрице должно быть равно количеству строк во второй. Результирующая матрица имеет количество строк первой и количество столбцов второй матрицы.

Помимо классического алгоритма умножения матриц, в данной лабораторной работе будут рассмотрены также алгоритм Копперсмита--Винограда и его оптимизированная версия.

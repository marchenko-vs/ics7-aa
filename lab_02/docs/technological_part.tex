\chapter{Технологическая часть}

В текущем разделе приведены средства реализации алгоритмов умножения матриц и листинги кода.

\section{Требования к программному обеспечению}

Программа должна запрашивать у пользователя размеры входных матриц, их элементы (целые числа), а также порядковый номер алгоритма умножения матриц. Затем на экран должен быть выведен результат умножения --- новая матрица.

Программа должна обрабатывать ошибки (например, попытку ввода отрицательного количества строк в матрице) и корректно завершать работу с выводом информации об ошибке на экран.
 
\section{Средства реализации}

Для реализации программного обеспечения был выбран язык C++ ввиду следующих причин:

\begin{enumerate}
\item[1)]
в библиотеке стандартных шаблонов имеется контейнер \textbf{std::vector}, который можно использовать для создания матриц;
\item[2)]
есть возможность определять свои структуры данных с помощью ключевого слова \textbf{struct};
\item[3)]
для считывания данных и вывода их на экран в стандартной библиотеке существуют соответственно функции \textbf{scanf()} и \textbf{printf()}.
\end{enumerate}

Таким образом, с помощью языка C++ можно реализовать программное обеспечение, которое соответствует перечисленным выше требованиям.

\section{Реализация алгоритмов}

В листинге \ref{lst:matrix.cpp} показана реализация матрицы.

\includelistingpretty
{matrix.cpp}{c++}{Реализация матрицы}

\subsection{Классический алгоритм}

В листинге \ref{lst:default_mul.cpp} показана реализация классического алгоритма умножения матриц.

\includelistingpretty
{default_mul.cpp}{c++}{Реализация классического алгоритма умножения матриц}

\subsection{Алгоритм Копперсмита--Винограда}

В листинге \ref{lst:winograd.cpp} показана реализация алгоритма Копперсмита--Винограда.

\includelistingpretty
{winograd.cpp}{c++}{Реализация алгоритма Копперсмита--Винограда}

\subsection{Оптимизированный алгоритм Копперсмита--Винограда}

В листинге \ref{lst:optimized_winograd.cpp} показана реализация оптимизированного алгоритма Копперсмита--Винограда.

\includelistingpretty
{optimized_winograd.cpp}{c++}{Реализация оптимизированного алгоритма Копперсмита--Винограда}

\section{Тестовые данные}

В таблице \ref{tabular:testsdata} приведены тестовые данные для трех функций, реализующих алгоритмы умножения матриц. 

Описание тестов:

\begin{enumerate}
\item[1)]
пустые входные матрицы;
\item[2)]
входные матрицы состоят из одного элемента;
\item[3)]
квадратные входные матрицы, количество столбцов первой матрицы четное;
\item[4)]
квадратные входные матрицы, количество столбцов первой матрицы нечетное;
\item[5)]
неквадратные входные матрицы.
\end{enumerate}

Тесты выполнялись по методологии черного ящика (модульное тестирование). Все тесты пройдены успешно.

\begin{table}[H]
\caption{Тестовые данные}
\label{tabular:testsdata}
\begin{tabular}{|p{4.5cm}|p{4.5cm}|p{6.8cm}|}
\hline
\textbf{Первая матрица} & \textbf{Вторая матрица} & \textbf{Выходная матрица} \tabularnewline
\hline
NULL & NULL & NULL \tabularnewline
\hline
\begin{equation*}
\begin{bmatrix}
5
\end{bmatrix}
\end{equation*} 
&
\begin{equation*}
\begin{bmatrix}
-9
\end{bmatrix}
\end{equation*} 
&
\begin{equation*}
\begin{bmatrix}
-45
\end{bmatrix}
\end{equation*} \tabularnewline
\hline
\begin{equation*}
\begin{bmatrix}
1 & 2 & 3 & 4 \\
5 & 6 & 7 & 8 \\
9 & 10 & 11 & 12 \\
13 & 14 & 15 & 16
\end{bmatrix}
\end{equation*}
&
\begin{equation*}
\begin{bmatrix}
1 & 2 & 3 & 4 \\
5 & 6 & 7 & 8 \\
9 & 10 & 11 & 12 \\
13 & 14 & 15 & 16
\end{bmatrix}
\end{equation*} 
& 
\begin{equation*}
\begin{bmatrix}
90 & 100 & 110 & 120 \\
202 & 228 & 254 & 280 \\
314 & 356 & 398 & 440 \\
426 & 484 & 542 & 600
\end{bmatrix}
\end{equation*} \tabularnewline
\hline
\begin{equation*}
\begin{bmatrix}
1 & 2 & 3 & 4 & 5 \\
6 & 7 & 8 & 9 & 10 \\
11 & 12 & 13 & 14 & 15 \\
16 & 17 & 18 & 19 & 20 \\
21 & 22 & 23 & 24 & 25 \\
\end{bmatrix}
\end{equation*}
&
\begin{equation*}
\begin{bmatrix}
1 & 2 & 3 & 4 & 5 \\
6 & 7 & 8 & 9 & 10 \\
11 & 12 & 13 & 14 & 15 \\
16 & 17 & 18 & 19 & 20 \\
21 & 22 & 23 & 24 & 25 \\
\end{bmatrix}
\end{equation*} 
& 
\begin{equation*}
\begin{bmatrix}
215 & 230 & 245 & 260 & 275 \\
490 & 530 & 570 & 610 & 650 \\
765 & 830 & 895 & 960 & 1025 \\
1040 & 1130 & 1220 & 1310 & 1400 \\
1315 & 1430 & 1545 & 1660 & 1775
\end{bmatrix}
\end{equation*} \tabularnewline
\hline
\begin{equation*}
\begin{bmatrix}
1 & 2 & 3 & 4 & 5 \\
6 & 7 & 8 & 9 & 10 \\
11 & 12 & 13 & 14 & 15 \\
16 & 17 & 18 & 19 & 20 \\
21 & 22 & 23 & 24 & 25 \\
26 & 28 & 28 & 29 & 30
\end{bmatrix}
\end{equation*}
&
\begin{equation*}
\begin{bmatrix}
1 & 2 & 3 & 4 \\
5 & 6 & 7 & 8 \\
9 & 10 & 11 & 12 \\
13 & 14 & 15 & 16 \\
17 & 18 & 19 & 20
\end{bmatrix}
\end{equation*} 
& 
\begin{equation*}
\begin{bmatrix}
175 & 190 & 205 & 220 \\
400 & 440 & 480 & 520 \\
625 & 690 & 755 & 820 \\
850 & 940 & 1030 & 1120 \\
1075 & 1190 & 1305 & 1420 \\
1300 & 1440 & 1580 & 1720
\end{bmatrix}
\end{equation*} \tabularnewline
\hline
\end{tabular}
\end{table}

\section*{Вывод из технологической части}

В данном разделе был написан исходный код трех алгоритмов умножения матриц --- классического, Копперсмита--Винограда и оптимизированного алгоритма Копперсмита--Винограда. Описаны тесты и приведены результаты тестирования.

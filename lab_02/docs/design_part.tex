\chapter{Конструкторская часть}
\section{Описание алгоритмов}

На рисунках~\ref{img:classic}--\ref{img:optimized_winograd_2} показаны схемы трех алгоритмов умножения матриц.

\includeimage
    {classic}
    {f}
    {H}
    {1\textwidth}
    {Классический алгоритм умножения матриц}
    
\includeimage
    {winograd_1}
    {f}
    {H}
    {0.9\textwidth}
    {Алгоритм Копперсмита--Винограда --- ч. 1}
    
\includeimage
    {winograd_2}
    {f}
    {H}
    {0.9\textwidth}
    {Алгоритм Копперсмита--Винограда --- ч. 2}
    
\includeimage
    {optimized_winograd_1}
    {f}
    {H}
    {0.8\textwidth}
    {Оптимизированный алгоритм Копперсмита--Винограда --- ч. 1}
    
\includeimage
    {optimized_winograd_2}
    {f}
    {H}
    {0.9\textwidth}
    {Оптимизированный алгоритм Копперсмита--Винограда --- ч. 2}

\section{Модель вычислений}

Пусть $M$ --- количество строк в первой матрице, $N$ --- количество столбцов в первой матрице и количество строк во второй, а $L$ --- количество столбцов во второй матрице.

Для вычисления трудоемкости введена следующая модель вычислений.

\begin{enumerate}
\item
Операции $>>, <<, [], +, -, =, \pluseq, \minuseq, ==, <, >, <=, >=,  !=, ++, --$ имеют трудоемкость 1.

\item
Операции $*, \%, /$ имеют трудоемкость 2.

\item
Трудоемкость условного оператора рассчитывается следующим образом:
\begin{equation}
f_{if} = f_{\text{условия}} +
	\begin{cases}
    f_A,~\text{если условие выполняется}, \\
    f_B,~\text{иначе}.
    \end{cases}
\end{equation}

\item
Трудоемкость цикла рассчитывается как:
\begin{equation}
f_{loop} = f_{\text{инициализации}} + f_{\text{сравнения}} + N \cdot (f_{\text{тела}} + f_{\text{инкремента}} +  f_{\text{сравнения}}).
\end{equation}

\item
Трудоемкость вызова функции равна 0.
\end{enumerate}

\section{Трудоемкость алгоритмов}

\subsection{Классический алгоритм}

Проверка на равенство количества столбцов в первой матрице и количества строк во второй:
\begin{equation}
f_1 = 1.
\end{equation}

Второй вложенный цикл:
\begin{equation}
\begin{gathered}
f_2 = 2 + 9 \cdot N.
\end{gathered}
\end{equation}

Первый вложенный цикл:
\begin{equation}
\begin{gathered}
f_3 = 2 + L \cdot (6 + f_2) = 2 + 8 \cdot L + 9 \cdot N \cdot L.
\end{gathered}
\end{equation}

Внешний цикл:
\begin{equation}
\begin{gathered}
f_4 = 2 + M \cdot (2 + f_3) = 2 + 4 \cdot M + 8 \cdot M \cdot L + 9 \cdot M \cdot N \cdot L.
\end{gathered}
\end{equation}

Таким образом, трудоемкость классического алгоритма умножения матриц равна:
\begin{equation}
\begin{gathered}
f = f_1 + f_4 = 3 + 4 \cdot M + 8 \cdot M \cdot L + 9 \cdot M \cdot N \cdot L \approx O(9 \cdot M \cdot N \cdot L).
\end{gathered}
\end{equation}

Если при умножении обе входные матрицы квадратные и одинакового порядка, например, $N$, то трудоемкость классического алгоритма равна:
\begin{equation}
\begin{gathered}
f \approx O(9 \cdot N^3).
\end{gathered}
\end{equation}

\subsection{Алгоритм Копперсмита--Винограда}

Проверка на равенство количества столбцов в первой матрице и количества строк во второй:
\begin{equation}
f_1 = 1.
\end{equation}

Первый цикл:
\begin{equation}
f_2 = 2 + M \cdot (2 + 2 + 4 + \frac{N}{2} \cdot (4 + 15)) = 2 + 8 \cdot M + \frac{19}{2} \cdot M \cdot N.
\end{equation}

Второй цикл:
\begin{equation}
f_3 = 2 + L \cdot (2 + 2 + 4 + \frac{N}{2} \cdot (4 + 15)) = 2 + 8 \cdot L + \frac{19}{2} \cdot N \cdot L.
\end{equation}

Третий цикл:
\begin{equation}
\begin{gathered}
f_4 = 2 + M \cdot (2 + 2 + L \cdot (2 + 7 + 4 + \frac{N}{2} \cdot (4 + 26))) = \\ = 2 + 4 \cdot M + 13 \cdot M \cdot L + 15 \cdot M \cdot N \cdot L.
\end{gathered}
\end{equation}

Если $N$ нечетное:
\begin{equation}
f_5 = 2 + M \cdot (2 + 2 + L \cdot (2 + 14)) = 2 + 4 \cdot M + 16 \cdot M \cdot L.
\end{equation}

Два цикла в теле условного оператора:
\begin{equation}
f_6 = 3 +
	\begin{cases}
    2 + 4 \cdot M + 16 \cdot M \cdot L, \\
    0.
    \end{cases}
\end{equation}

Таким образом, трудоемкость алгоритма Копперсмита--Винограда при нечетном $N$:
\begin{equation}
\begin{gathered}
f = f_1 + f_2 + f_3 + f_4 + f_5 + f_6 = \\ = 12 + 16 \cdot M + 8 \cdot L + \frac{19}{2} \cdot M \cdot N + \frac{19}{2} \cdot N \cdot L + \\ + 29 \cdot M \cdot L + 15 \cdot M \cdot N \cdot L \approx O(15 \cdot M \cdot N \cdot L).
\end{gathered}
\end{equation}

Трудоемкость алгоритма Копперсмита--Винограда при четном $N$:
\begin{equation}
\begin{gathered}
f = f_1 + f_2 + f_3 + f_4 + f_5 + f_6 = \\ = 10 + 12 \cdot M + 8 \cdot L + \frac{19}{2} \cdot M \cdot N + \frac{19}{2} \cdot N \cdot L + \\ + 13 \cdot M \cdot L + 15 \cdot M \cdot N \cdot L \approx O(15 \cdot M \cdot N \cdot L).
\end{gathered}
\end{equation}

Если при умножении обе входные матрицы квадратные и одинакового порядка, например, $N$, то трудоемкость равна:
\begin{equation}
f \approx O(15 \cdot N^3).
\end{equation}

\subsection{Оптимизированный алгоритм Копперсмита--Винограда}

Проверка на равенство количества столбцов в первой матрице и количества строк во второй, а также инициализация переменной, равной $\frac{N}{2}$:
\begin{equation}
f_1 = 1 + 3 = 4.
\end{equation}

Первый цикл:
\begin{equation}
f_2 = 2 + M \cdot (2 + 2 + 2 + \frac{N}{2} \cdot (2 + 11)) = 2 + 6 \cdot M + \frac{13}{2} \cdot M \cdot N.
\end{equation}

Второй цикл:
\begin{equation}
f_3 = 2 + L \cdot (2 + 2 + 2 + \frac{N}{2} \cdot (2 + 11)) = 2 + 6 \cdot L + \frac{13}{2} \cdot N \cdot L.
\end{equation}

Третий цикл:
\begin{equation}
\begin{gathered}
f_4 = 2 + M \cdot (2 + 2 + L \cdot (2 + 7 + 2 + \frac{N}{2} \cdot (2 + 18))) = \\ = 2 + 4 \cdot M + 11 \cdot M \cdot L + 10 \cdot M \cdot N \cdot L.
\end{gathered}
\end{equation}

Если $N$ нечетное:
\begin{equation}
f_5 = 2 + M \cdot (2 + 2 + L \cdot (2 + 12)) = 2 + 4 \cdot M + 14 \cdot M \cdot L.
\end{equation}

Два цикла в теле условного оператора:
\begin{equation}
f_6 = 3 +
	\begin{cases}
    2 + 4 \cdot M + 14 \cdot M \cdot L, \\
    0.
    \end{cases}
\end{equation}

Таким образом, трудоемкость оптимизированного алгоритма Копперсмита--Винограда при нечетном $N$:
\begin{equation}
\begin{gathered}
f = f_1 + f_2 + f_3 + f_4 + f_5 + f_6 = \\ = 15 + 14 \cdot M + 6 \cdot L + \frac{13}{2} \cdot M \cdot N + \frac{13}{2} \cdot N \cdot L + \\ + 25 \cdot M \cdot L + 10 \cdot M \cdot N \cdot L \approx O(10 \cdot M \cdot N \cdot L).
\end{gathered}
\end{equation}

Трудоемкость оптимизированного алгоритма Копперсмита--Винограда при четном $N$:
\begin{equation}
\begin{gathered}
f = f_1 + f_2 + f_3 + f_4 + f_5 + f_6 = \\ = 13 + 10 \cdot M + 6 \cdot L + \frac{13}{2} \cdot M \cdot N + \frac{13}{2} \cdot N \cdot L + \\ + 11 \cdot M \cdot L + 10 \cdot M \cdot N \cdot L \approx O(10 \cdot M \cdot N \cdot L).
\end{gathered}
\end{equation}

Если при умножении обе входные матрицы квадратные и одинакового порядка, например, $N$, то трудоемкость равна:
\begin{equation}
f \approx O(10 \cdot N^3).
\end{equation}

\section*{Вывод из конструкторской части}

В текущем разделе на основе теоретических данных, полученных из аналитического раздела, были построены схемы трех алгоритмов умножения матриц --- классического, Копперсмита--Винограда и оптимизированного алгоритма Копперсмита--Винограда, а также оценены их трудоемкости.

Исходя из полученных данных, наиболее эффективным должен быть классический алгоритм умножения матриц, на втором месте --- оптимизированный алгоритм Копперсмита--Винограда, а на третьем --- его обычная версия.

\chapter{Аналитическая часть}

\section{Цели и задачи}

Цель работы: получить навыки оценки трудоемкости и временной эффективности на материале алгоритмов умножения матриц.

Задачи текущей лабораторной работы:

\begin{enumerate}
\item[1)]
изучить и реализовать три алгоритма умножения матриц --- классический, Копперсмита--Винограда и оптимизированный алгоритм Копперсмита--Винограда;
\item[2)]
провести сравнительный анализ трудоемкости алгоритмов на основе теоретических расчетов и выбранной модели вычислений;
\item[3)]
провести сравнительный анализ времени работы алгоритмов на основе экспериментальных данных.
\end{enumerate}

\section{Классический алгоритм}

Пусть даны две прямоугольные матрицы $A$ и $B$ размерности $m \times n$ и $n \times r$ соответственно:

\begin{equation}
A = \begin{bmatrix}
a_{11} & a_{12} & \hdots & a_{1n} \\
a_{21} & a_{22} & \hdots & a_{2n} \\
\vdots & \vdots & \ddots & \vdots \\
a_{m1} & a_{m2} & \hdots & a_{mn}
\end{bmatrix},
\end{equation}

\begin{equation}
B = \begin{bmatrix}
b_{11} & b_{12} & \hdots & b_{1r} \\
b_{21} & b_{22} & \hdots & b_{2r} \\
\vdots & \vdots & \ddots & \vdots \\
b_{n1} & b_{n2} & \hdots & b_{nr}
\end{bmatrix}.
\end{equation}


Произведение матрицы $A \equiv [a_{ij}]$ размера $m \times n$ на матрицу $B \equiv [b_{jk}]$ размера $n \times r$ есть матрица $C \equiv [c_{ik}]$ размера $m \times r$

\begin{equation}
C = AB \equiv [a_{ij}][b_{jk}] \equiv [c_{ik}],
\end{equation}

где

\begin{equation}
c_{ik} = \sum\limits_{j=1}^n a_{ij}b_{jk}.
\end{equation}

Таким образом, элемент $c_{ik}$ матрицы $C = AB$ есть сумма произведений элементов $i$-й строки матрицы $A$ на соответствующие элементы $k$-го столбца матрицы $B$ \cite{Korn1973}.

Операция умножения двух матриц выполнима только в том случае, если число столбцов в первом сомножителе равно числу строк во втором. В частности, умножение всегда выполнимо, если оба сомножителя --- квадратные матрицы одного и того же порядка.

\section{Алгоритм Копперсмита--Винограда}

Если посмотреть на результат умножения двух матриц, то видно, что каждый элемент в нем представляет собой скалярное произведение соответствующих строки и столбца исходных матриц. Можно заметить также, что такое умножение допускает предварительную обработку, позволяющую часть работы выполнить заранее.

Рассмотрим два вектора $A = (a_1, a_2, a_3, a_4)$ и $B = (b_1, b_2, b_3, b_4)$. Их скалярное произведение равно:
\begin{equation}
A \cdot B = a_1 \cdot b_1 + a_2 \cdot b_2 + a_3 \cdot b_3 + a_4 \cdot b_4.
\end{equation}
Это равенство можно переписать в виде:
\begin{equation}
A \cdot B = (a_1 + b_2)(a_2 + b_1) + (a_3 + b_4)(a_4 + b_3) - a_1 \cdot a_2 - a_3 \cdot a_4 - b_1 \cdot b_2 - b_3 \cdot b_4.
\end{equation}

Выражение в правой части последнего равенства допускает предварительную обработку: его части можно вычислить заранее и запомнить для каждой строки первой матрицы и для каждого столбца второй.

\section{Оптимизированный алгоритм Копперсмита--Винограда}

Для оптимизации описанного в предыдущем пункте алгоритма Копперсмита--Винограда операция $x = x + k$ будет заменена на $x \pluseq k$, умножение на 2 будет заменено побитовым сдвигом влево, а некоторые слагаемые, используемые в алгоритме, будут вычисляться заранее.

\section*{Вывод из аналитической части}

В текущем разделе были рассмотрены три алгоритма умножения матриц: классический, Копперсмита--Винограда и оптимизированный алгоритм Копперсмита--Винограда.

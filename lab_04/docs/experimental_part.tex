\chapter{Исследовательская часть}

\section{Технические характеристики устройства}

Технические характеристики устройства, на котором было проведено измерение времени работы алгоритмов:

\begin{enumerate}
\item[1)]
операционная система Windows 11 Pro x64;
\item[2)]
оперативная память 16 ГБ;
\item[3)]
процессор Intel\textregistered ~Core\texttrademark ~i7-4790K @ 4.00 ГГц.
\end{enumerate}

\section{Время работы алгоритмов}

Время работы функции замерено с помощью ассемблерной инструкции \textbf{rdtsc}, которая читает счетчик Time Stamp Counter и возвращает 64-битное количество тиков с момента последнего сброса процессора.

В таблице~\ref{tabular:time} приведено время работы в тиках функции, реализующей алгоритм сортировки слиянием, в зависимости от количества потоков при различном количестве элементов во входном массиве.
На рисунке~\ref{img:graph} изображена зависимость времени работы в тиках функции, реализующей алгоритм сортировки слиянием, от количества потоков при различном количестве элементов во входном массиве.

\begin{table}[H]
\caption{Время работы в тиках алгоритма сортировки слиянием в зависимости от количества потоков при различном количестве элементов во входном массиве}
\label{tabular:time}
\begin{tabular}{|>{\raggedleft}p{3cm}|>{\raggedleft}p{1.5cm}|>{\raggedleft}p{1.5cm}|>{\raggedleft}p{1.5cm}|>{\raggedleft}p{1.5cm}|>{\raggedleft}p{1.5cm}|>{\raggedleft}p{1.5cm}|>{\raggedleft}p{1.5cm}|>{\raggedleft}p{1.5cm}|}
\hline
\textbf{Количество потоков} & \textbf{100} & \textbf{200} & \textbf{300} & \textbf{400} & \textbf{500} & \textbf{600} & \textbf{700} 
\tabularnewline
\hline
1 & 15493 & 15512 & 17082 & 17175 & 17337 & 17976 & 18384
\tabularnewline
\hline
2 & 53227 & 70352 & 104176 & 120203 & 122818 & 153112 & 188254
\tabularnewline
\hline
3 & 99208 & 101051 & 137593 & 157357 & 201546 & 216977 & 222431
\tabularnewline
\hline
4 & 100561 & 137027 & 160311 & 205088 & 234043 & 288583 & 338209
\tabularnewline
\hline
5 & 102323 & 152047 & 203392 & 245139 & 272109 & 316746 & 343309
\tabularnewline
\hline
6 & 109825 & 179446 & 225440 & 260116 & 314463 & 416150 & 424402
\tabularnewline
\hline
7 & 138963 & 172710 & 288395 & 297874 & 348914 & 446906 & 470084
\tabularnewline
\hline
8 & 137146 & 214064 & 240727 & 314933 & 383970 & 440488 & 497007
\tabularnewline
\hline
\end{tabular}
\end{table}

\includeimage
    {graph}
    {f}
    {H}
    {1\textwidth}
    {Зависимость времени работы в тиках функции, реализующей алгоритм сортировки слиянием, от количества потоков при различном количестве элементов во входном массиве}

\section*{Вывод из исследовательской части}

Согласно полученным при проведении эксперимента данным, быстрее всего работает последовательный алгоритм сортировки сляинием. 
Это объясняется тем, что при увеличении количества потоков в конце работы алгоритма нужно соединять большее количество массивов. 
Таким образом, наиболее эффективным можно считать однопоточный алгоритм сортировки сляинием.

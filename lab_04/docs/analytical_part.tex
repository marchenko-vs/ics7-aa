\chapter{Аналитическая часть}

\section{Цели и задачи}

Цель работы: изучить параллельные вычисления и реализовать алгоритм сортировки слиянием с использованием параллельных вычислений.

Задачи лабораторной работы:

\begin{enumerate}
\item[1)]
изучить понятие параллельных вычислений;
\item[2)]
реализовать последовательный алгоритм сортировки слиянием;
\item[3)]
реализовать алгоритм сортировки слиянием с использованием параллельных вычислений;
\item[4)]
провести сравнительный анализ времени работы алгоритмов на основе экспериментальных данных.
\end{enumerate}

\section{Алгоритм сортировки слиянием}

Сортировка слиянием --- алгоритм сортировки, который упорядочивает массив в определенном порядке. 
Эта сортировка использует принцип <<разделяй и властвуй>> \cite{Cormen2011}. 
Сначала массив делится на несколько подмассивов меньшего размера. 
Затем эти массивы сортируются с помощью рекурсивного вызова. 
Наконец, все подмассивы соединяются в один, и получается решение исходной задачи.

Этапы сортировки массива слиянием выглядят следующим образом.

\begin{enumerate}
\item
Сортируемый массив делится на две части примерно одинакового размера.
\item
Каждая из получившихся частей сортируется отдельно --- рекурсивно с помощью этого же алгоритма сортировки.
\item
Два упорядоченных массива соединяются в один.
\end{enumerate}

Притом рекурсивное разбиение задачи на подзадачи происходит до тех пор, пока размер подмассива не достигнет единицы (любой массив длины 1 можно считать упорядоченным). 

\section*{Вывод из аналитической части}

В текущем разделе был рассмотрен алгоритм сортировки слиянием. 
Этот алгоритм хорошо подходит для внедрения параллельных вычислений, поскольку во время работы он выделяет подмассивы, что позволяет сортировать их одновременно, используя несколько потоков.

{\center\chapter*{Введение}}
\addcontentsline{toc}{chapter}{Введение}

Параллелизм --- это возможность одновременного выполнения более одной арифметико-логической операции или программной ветви \cite{Shpakovskiy2013}. 
Параллельная обработка данных, воплощая идею одновременного выполнения нескольких действий, имеет две разновидности: конвейерность и собственно параллельность \cite{Antonov2002}.

Параллельная обработка. 
Если некое устройство выполняет одну операцию за единицу времени, то тысячу операций оно выполнит за тысячу единиц. 
Если предположить, что имеется пять таких же независимых устройств, способных работать одновременно и независимо, то ту же тысячу операций система из пяти устройств может выполнить уже не за тысячу, а за двести единиц времени. 
Аналогично, система из $N$ устройств ту же работу выполнит примерно за $\frac{1000}{N}$ единиц времени \cite{Antonov2002}.

Конвейерная обработка. 
Идея конвейерной обработки заключается в выделении отдельных этапов выполнения общей операции, причем каждый этап, выполнив свою работу, передает результат следующему, одновременно принимая новую порцию входных данных \cite{Antonov2002}.

В рамках данной лабораторной работы параллельные вычисления будут исследоваться на материале алгоритмов сортировки.

Задача сортировки формально определяется следующим образом.

Вход: последовательность из $n$ чисел $\langle a_1, a_2,...,a_n \rangle$.

Выход: перестановка (изменение порядка) $\langle a_1', a_2',...,a_n' \rangle$ входной последовательности таким образом, что для ее членов выполняется соотношение $a_1' \leq a_2'\leq ... \leq a_n'$ \cite{Cormen2011}.

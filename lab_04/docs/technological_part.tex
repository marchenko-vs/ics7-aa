\chapter{Технологическая часть}

В текущем разделе приведены средства реализации алгоритма сортировки слиянием и объединения массивов и листинги кода.

\section{Требования к программному обеспечению}

Программа должна запрашивать у пользователя размер массива и его элементы (целые числа). 
Затем на экран должен быть выведен результат --- отсортированный в порядке возрастания массив.

Программа должна обрабатывать ошибки (например, попытку ввода отрицательного размера массива) и корректно завершать работу с выводом информации об ошибке на экран.
 
\section{Средства реализации}

Для реализации программного обеспечения был выбран язык \textbf{C} ввиду следующих причин:

\begin{enumerate}
\item[1)]
в языке есть возможность создания статических и динамических массивов;
\item[2)]
есть возможность создавать потоки с помощью библиотечного вызова \textbf{pthread\_create()};
\item[3)]
для считывания данных и вывода их на экран в стандартной библиотеке существуют соответственно функции \textbf{scanf()} и \textbf{printf()}.
\end{enumerate}

Таким образом, с помощью языка \textbf{C} можно реализовать программное обеспечение, которое соответствует перечисленным выше требованиям.

\section{Реализация алгоритмов}

\subsection{Объединение массивов}

В листинге~\ref{lst:merge.c} показана реализация алгоритма объединения массивов.

\includelistingpretty
{merge.c}{c}{Реализация алгоритма объединения массивов}

\subsection{Сортировка слиянием}

В листинге~\ref{lst:merge_sort.c} показана реализация последовательного алгоритма сортировки слиянием.

\includelistingpretty
{merge_sort.c}{c++}{Реализация последовательного алгоритма сортировки слиянием}

\subsection{Сортировка слиянием с использованием параллельных вычислений}

В листинге~\ref{lst:thread.c} показана реализация вспомогательной структуры для использования потоков.

\includelistingpretty
{thread.c}{c}{Реализация вспомогательной структуры для использования потоков}

В листинге~\ref{lst:thread_merge_sort.c} показана реализация алгоритма сортировки слиянием с использованием параллельных вычислений.

\includelistingpretty
{thread_merge_sort.c}{c++}{Реализация алгоритма сортировки слиянием с использованием параллельных вычислений}

\section{Тестовые данные}

В таблице~\ref{tabular:testsdata} приведены тестовые данные для функции, реализующей алгоритм сортировки слиянием. 

Тесты выполнялись по методологии черного ящика (модульное тестирование). Все тесты пройдены успешно.

\begin{table}[H]
\caption{Тестовые данные}
\label{tabular:testsdata}
\begin{tabular}{|p{3.5cm}|p{4cm}|p{4cm}|p{4cm}|}
\hline
\textbf{Описание теста} & \textbf{Входной массив} & \textbf{Ожидаемый результат} & \textbf{Выходной массив}
\tabularnewline
\hline
Пустой массив & NULL & NULL & NULL
\tabularnewline
\hline
Один элемент в массиве & -25 & -25 & -25
\tabularnewline
\hline
Упорядоченный по возрастанию массив & -282, -50, 32, 54, 76, 108 & -282, -50, 32, 54, 76, 108 & -282, -50, 32, 54, 76, 108
\tabularnewline
\hline
Упорядоченный по убыванию массив & 982, 654, 54, 3, -19, -89, -320 & -320, -89, -19, 3, 54, 654, 982 & -320, -89, -19, 3, 54, 654, 982
\tabularnewline
\hline
Случайно упорядоченный массив & 10, -20, 32, -89, -76, -10, 89, 35, 197 & -89, -76, -20, -10, 10, 32, 35, 89, 197 & -89, -76, -20, -10, 10, 32, 35, 89, 197 \tabularnewline
\hline
\end{tabular}
\end{table}

\section*{Вывод из технологической части}

В данном разделе был написан исходный код алгоритмов объединения массивов и сортировки слиянием (последовательного и с использованием параллельных вычислений). 
Описаны тесты и приведены результаты тестирования.

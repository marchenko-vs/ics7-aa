\chapter{Исследовательская часть}

\section{Технические характеристики устройства}

Технические характеристики устройства, на котором было проведено измерение времени работы алгоритмов:

\begin{enumerate}
\item[1)]
операционная система Windows 11 Pro x64;
\item[2)]
оперативная память 16 ГБ;
\item[3)]
процессор Intel\textregistered ~Core\texttrademark ~i7-4790K @ 4.00 ГГц.
\end{enumerate}

\pagebreak
\section{Время работы алгоритмов}

Время работы функции замерено с помощью ассемблерной инструкции \textbf{rdtsc}, которая читает счетчик Time Stamp Counter и возвращает 64-битное количество тиков с момента последнего сброса процессора.

В таблице~\ref{tabular:time} приведено время работы в тиках линейной и конвейерной обработки заявок в зависимости от количества заявок.
На рисунке~\ref{img:graph} изображена зависимость времени работы в тиках функций, реализующих линейную и конвейерную обработку заявок в зависимости от количества заявок.

\begin{table}[H]
\caption{Время работы в тиках линейной и конвейерной обработки заявок}
\label{tabular:time}
\begin{tabular}{|>{\raggedleft}p{3.5cm}|>{\raggedleft}p{6cm}|>{\raggedleft}p{6cm}|}
\hline
\textbf{Количество заявок} & \textbf{Линейная обработка} & \textbf{Конвейерная обработка} 
\tabularnewline
\hline
10 & 167304 & 471175
\tabularnewline
\hline
20 & 298027 & 544363
\tabularnewline
\hline
30 & 435152 & 546690
\tabularnewline
\hline
40 & 523112 & 556768
\tabularnewline
\hline
50 & 570023 & 601194
\tabularnewline
\hline
60 & 686216 & 657707
\tabularnewline
\hline
70 & 786345 & 680234
\tabularnewline
\hline
80 & 800943 & 710394
\tabularnewline
\hline
90 & 1000324 & 897345
\tabularnewline
\hline
100 & 1382930 & 987234
\tabularnewline
\hline
\end{tabular}
\end{table}

\includeimage
    {graph}
    {f}
    {H}
    {1\textwidth}
    {Зависимость времени работы в тиках функций, реализующих линейную и конвейерную обработку заявок в зависимости от количества заявок}
    
\section{Пример файла журнала}

В листинге~\ref{lst:log.sh} показан пример файла журнала, который генерируется программой после обработки заявок.

\includelistingpretty
{log.sh}{sh}{Пример файла журнала}

\section{Вывод из исследовательской части}

В данном разделе был проведен эксперимент по измерению времени работы линейной и конвейерной обработки заявок. 
Согласно полученным при проведении эксперимента данным, конвейерная обработка данных начинает работать быстрее линейной при количестве входных заявок больше 50.
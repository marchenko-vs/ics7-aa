{\center\chapter*{Заключение}}
\addcontentsline{toc}{chapter}{Заключение}

В рамках данной лабораторной работы была достигнута поставленная цель: были изучены принципы конвейерной обработки данных и реализованы классический алгоритм поиска строки в тексте и алгоритм Кнута-Морриса-Пратта.

Решены все поставленные задачи:

\begin{enumerate}
\item[1)] изучено понятие конвейерной обработки данных;
\item[2)] реализованы классический алгоритм поиска строки в тексте и алгоритм Кнута-Морриса-Пратта;
\item[3)] реализована последовательная обработка данных на основе двух алгоритмов поиска строки в тексте;
\item[4)] реализована конвейерная обработка данных на основе двух алгоритмов поиска строки в тексте;
\item[5)] проведен сравнительный анализ времени работы линейной и конвейерной обработки данных на основе экспериментальных данных.
\end{enumerate}

В ходе проведения эксперимента был сделан вывод, что конвейерная обработка данных начинает работать быстрее при количестве входных заявок больше 50.

\chapter{Конструкторская часть}

\section{Описание алгоритмов}

\subsection{Алгоритм Кнута-Морриса-Пратта}

На рисунке~\ref{img:lps} показана схема алгоритма заполнения вспомогательного массива.

\includeimage
    {lps}
    {f}
    {H}
    {0.9\textwidth}
    {Алгоритм заполнения вспомогательного массива}

На рисунках~\ref{img:kmp_1}--\ref{img:kmp_2} показана схема алгоритма Кнута-Морриса-Пратта.

\includeimage
    {kmp_1}
    {f}
    {H}
    {0.7\textwidth}
    {Алгоритм Кнута-Морриса-Пратта --- ч. 1}
        
\includeimage
    {kmp_2}
    {f}
    {H}
    {0.9\textwidth}
    {Алгоритм Кнута-Морриса-Пратта --- ч. 2}

\pagebreak
\subsection{Классический алгоритм поиска строки в тексте}

На рисунке~\ref{img:default} показана схема классического алгоритма поиска строки в тексте. 

\includeimage
    {default}
    {f}
    {H}
    {0.7\textwidth}
    {Классический алгоритм поиска строки в тексте}

\section*{Вывод из конструкторской части}

В текущем разделе на основе теоретических данных, полученных из аналитического раздела, были построены схемы двух алгоритмов поиска строки в тексте --- классического и Кнута-Морриса-Пратта.

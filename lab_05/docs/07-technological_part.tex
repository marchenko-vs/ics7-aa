\chapter{Технологическая часть}

В текущем разделе приведены средства реализации двух алгоритмов поиска строки в тексте, алгоритмов последовательной и конвейерной обработки данных и листинги кода.

\section{Требования к программному обеспечению}

На вход программе передается два аргумента командной строки --- название файла, содержащего текст, и название файла, в котором записано количество строк и сами строки. 
Затем программа обрабатывает заявки с помощью последовательной и конвейерной обработки данных. 
В результате работы программы создаются два файла (журнала), в которых записаны результаты обработки заявок.

Программа должна обрабатывать ошибки (например, отсутствие файла с названием, переданным в качестве аргумента командной строки) и корректно завершать работу с выводом информации об ошибке на экран.
 
\section{Средства реализации}

Для реализации программного обеспечения был выбран язык \textbf{C\texttt{++}} ввиду следующих причин:

\begin{enumerate}
\item[1)] есть возможность создания массивов символов;
\item[2)] есть возможность создания потоков с помощью класса \textbf{std::thread};
\item[3)] для считывания данных из файла в стандартной библиотеке реализованы функции \textbf{fscanf()} и \textbf{fgets()};
\item[4)] для записи данных в файл в стандартной библиотеке реализована функция \textbf{fprintf()};
\item[5)] есть возможность принимать аргументы командной строки.
\end{enumerate}

Таким образом, с помощью языка \textbf{C\texttt{++}} можно реализовать программное обеспечение, которое соответствует перечисленным выше требованиям.

\section{Реализация алгоритмов}

\subsection{Классический алгоритм поиска строки в тексте}

В листинге~\ref{lst:default.cpp} показана реализация классического алгоритма поиска строки в тексте.

\includelistingpretty
{default.cpp}{c++}{Реализация классического алгоритма поиска строки в тексте}

\subsection{Алгоритм Кнута-Морриса-Пратта}

В листинге~\ref{lst:kmp.cpp} показана реализация алгоритма Кнута-Морриса-Пратта.

\includelistingpretty
{kmp.cpp}{c++}{Реализация алгоритма Кнута-Морриса-Пратта}

\subsection{Алгоритм последовательной обработки данных}

В листинге~\ref{lst:linear.cpp} показана реализация последовательной обработки данных.

\includelistingpretty
{linear.cpp}{c++}{Реализация последовательной обработки данных}

\subsection{Алгоритм конвейерной обработки данных}

В листинге~\ref{lst:structs.cpp} показана реализация двух структур данных --- заявки и состояния заявки.

\includelistingpretty
{structs.cpp}{c++}{Реализация структур данных}

В листинге~\ref{lst:conveyor.cpp} показана реализация конвейерной обработки данных.

\includelistingpretty
{conveyor.cpp}{c++}{Реализация конвейерной обработки данных}

\section{Тестовые данные}

В таблице~\ref{tabular:testsdata} приведены тестовые данные для двух функций, реализующих соответственно классический алгоритм поиска строки в тексте и алгоритм Кнута-Морриса-Пратта. 

Тесты выполнялись по методологии черного ящика (модульное тестирование). Все тесты пройдены успешно.

\begin{table}[H]
\caption{Тестовые данные}
\label{tabular:testsdata}
\begin{tabular}{|p{4.5cm}|p{4cm}|p{3.5cm}|p{3.5cm}|}
\hline
\textbf{Текст} & \textbf{Строка} & \textbf{Классический алгоритм} & \textbf{Алгоритм Кнута-Морриса-Пратта}
\tabularnewline
\hline
NULL & NULL & -1 & -1
\tabularnewline
\hline
ababcabcbabcbcbab & ababcabcbabcbcbab & 0 & 0
\tabularnewline
\hline
ababcabcbabcbcbab & q & -1 & -1
\tabularnewline
\hline
ababcabcbabcbcbab & ab & 0, 2, 5, 9, 15 & 0, 2, 5, 9, 15
\tabularnewline
\hline
abababababa\-babababab & aba & 0, 2, 4, 6, 8, 10, 12, 14, 16, 18 & 0, 2, 4, 6, 8, 10, 12, 14, 16, 18
\tabularnewline
\hline
\end{tabular}
\end{table}

\section*{Вывод из технологической части}

В данном разделе был написан исходный код двух алгоритмов поиска строки в тексте --- классического и Кнута-Морриса-Пратта. 
Также написан исходный код последовательной и конвейерной обработки данных. 
Описаны тесты и приведены результаты тестирования.

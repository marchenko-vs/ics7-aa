\chapter{Технологическая часть}

В текущем разделе приведены средства реализации двух алгоритмов решения задачи коммивояжера.

\section{Требования к программному обеспечению}

Программа должна предоставлять пользователю три функции:
\begin{enumerate}
\item[1)] поиск гамильтонова цикла с помощью алгоритма полного перебора;
\item[2)] поиск гамильтонова цикла с помощью муравьиного алгоритма;
\item[3)] параметризация муравьиного алгоритма.
\end{enumerate}

Притом для решения задачи коммивояжера используется файл с матрицей расстояний, имя которого пользователь должен ввести по запросу программы.

Программа должна обрабатывать ошибки (например, отсутствие файла с матрицей расстояний) и корректно завершать работу с выводом информации об ошибке на экран.
 
\section{Средства реализации}

Для реализации программного обеспечения был выбран язык \textbf{Python} ввиду следующих причин:
\begin{enumerate}
\item[1)] существует модуль \textbf{numpy}, в котором реализован класс \textbf{array} для работы с массивами и матрицами;
\item[2)] для считывания данных из файла реализован метод \textbf{readline()};
\item[3)] для записи данных в файл реализован метод \textbf{write()}.
\end{enumerate}

Таким образом, с помощью языка \textbf{Python} можно реализовать программное обеспечение, которое соответствует перечисленным выше требованиям.

\section{Реализация алгоритмов}

\subsection{Алгоритм полного перебора}

В листинге~\ref{lst:tsp.py} показана реализация алгоритма полного перебора для решения задачи коммивояжера.

\includelistingpretty
{tsp.py}{python}{Реализация алгоритма полного перебора}

\pagebreak
\subsection{Муравьиный алгоритм}

В листинге~\ref{lst:aco.py} показана реализация муравьиного алгоритма для решения задачи коммивояжера.

\includelistingpretty
{aco.py}{python}{Реализация муравьиного алгоритма}

\section{Тестовые данные}

В таблице~\ref{tabular:testsdata} приведены тестовые данные для двух функций, реализующих алгоритмы для решения задачи коммивояжера (поиска гамильтонова цикла). 
Результата записаны в следующем формате: значение кратчайшего пути; кратчайший путь.

Тесты выполнялись по методологии черного ящика (модульное тестирование). 
Все тесты пройдены успешно.

\begin{table}[H]
\caption{Тестовые данные}
\label{tabular:testsdata}
\begin{tabular}{|p{6cm}|p{5cm}|p{4cm}|}
\hline
\textbf{Матрица расстояний} & \textbf{Алгоритм полного перебора} & \textbf{Муравьиный алгоритм}
\tabularnewline
\hline
\begin{equation*}
\begin{bmatrix}
0
\end{bmatrix}
\end{equation*} & 0; [1, 1] & 0; [1, 1]
\tabularnewline
\hline
\begin{equation*}
\begin{bmatrix}
0 & 10 \\
10 & 0 \\
\end{bmatrix}
\end{equation*} & 10; [1, 2, 1] & 10; [1, 2, 1]
\tabularnewline
\hline
\begin{equation*}
\begin{bmatrix}
0 & 10 & 15 & 20 \\
10 & 0 & 35 & 25 \\
15 & 35 & 0 & 30 \\
20 & 25 & 30 & 0 \\
\end{bmatrix}
\end{equation*} & 80; [1, 2, 4, 3, 1] & 80; [1, 2, 4, 3, 1]
\tabularnewline
\hline
\end{tabular}
\end{table}

\section*{Вывод из технологической части}

В текущем разделе был написан исходный код алгоритма полного перебора и муравьиного алгоритма для решения задачи коммивояжера. 
Описаны тесты и приведены результаты тестирования.
